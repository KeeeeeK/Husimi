\documentclass[a4paper, 12pt]{article}

%Русский язык
\renewcommand{\familydefault}{\sfdefault}%шрифт
\usepackage[T2A]{fontenc} %кодировка
\usepackage[utf8]{inputenc} %кодировка исходного кода
\usepackage[english,russian]{babel} %локализация и переносы
%отступы 
\usepackage[left=2cm,right=2cm,top=2cm,bottom=3cm,bindingoffset=0cm]{geometry}
\usepackage{indentfirst}
%Вставка картинок
\usepackage{graphicx}
\graphicspath{}
\DeclareGraphicsExtensions{.pdf,.png,.jpg, .jpeg}

%Таблицы
\usepackage[table,xcdraw]{xcolor}
\usepackage{booktabs}

% Cсылки
\usepackage{hyperref}
%Математика
\usepackage{amsmath, amsfonts, amssymb, amsthm, mathtools }
\newenvironment{eqw}{\begin{equation} \begin{aligned}}   
    {\end{aligned}    \end{equation}}
\newenvironment{eqw*}{\begin{equation*} \begin{aligned}}   
    {\end{aligned}    \end{equation*}}
\DeclareMathOperator*{\Res}{Res}
\DeclareMathOperator*{\sign}{sign}
\DeclareMathOperator*{\Real}{Re}
\DeclareMathOperator*{\Imag}{Im}
%Заголовок
\author{Нугманов Булат}
\title{Нахождение интеграла для $F$}
\begin{document}
\maketitle
\section*{Разложение $z_k$}
\subsection*{Разложение функции Ламберта из статьи}
Следующая формула взята из статьи "On the Lambert W Function"(DOI:10.1007/BF02124750), формула (4.20):
\begin{equation}
\begin{aligned}
    W_k(z) = \log z + 2\pi i k - \log\left(\log z + 2\pi i k\right) + \sum\limits_{k=0}^{\infty}\sum\limits_{m=1}^{\infty} c_{km}\log^m\left(\log z + 2\pi i k\right)\left(\log z + 2\pi i k \right)^{-k-m}
\end{aligned}
\end{equation}

Для того, чтоб ветви функции Ламберта совпадали с общепринятыми, ветви $\log z$ необходимо так же брать привычными --- с разрезом на отрицательных числах и нулевой мнимой частью при положительных $z$. Коэффициенты $c_{km}$ определены в статье после формулы (4.18):
\begin{equation}
\begin{aligned}
    c_{km} = \frac{(-1)^k}{m!} c(k+m, k+1)
\end{aligned}
\end{equation}

$c(k+m, k+1)$ --- это беззнаковые числа Стирлинга первого рода. В вольфраме они обозначаются как "Abs@StirlingS1[k+m, k+1]".

В нашей же задаче, требуется определить $z_k = \frac{i}{2}W_k(-2i\alpha\gamma)$. Обозначая $z=-2i\alpha\gamma$, $k+m=n$, получаем:
\begin{equation}
\begin{aligned}
    -2iz_k &= W_k(z) \\
    &= \log z + 2\pi i k - \log\left(\log z + 2\pi i k\right) + \dots \\
    &\dots + \sum\limits_{n=1}^{\infty}\sum\limits_{m=1}^{n} 
    \frac{(-1)^{n-m}}{m!} c(n, n-m+1)\frac{\log^{m}\left(\log z + 2\pi i k\right)}{\left(\log z + 2\pi i k\right)^n}
\end{aligned}
\end{equation}
В такой форме наглядно видно разложение по малости остаточных членов. В дальнейшем будет видно, что большим параметром при разложении здесь является номер функции Ламберта --- $k$. Ещё можно использовать знаковые числа Стирлинга ($s(n, k) = (-1)^{n-k}c(n, k) \Rightarrow (-1)^{n-m} c(n, n-m+1) = (-1)^{n+1} s(n, n-m+1)$), однако в этом нет пока необходимости.

\section*{Метод перевала с остаточными членами}
\subsection*{Общая теория метода перевала}
Следует быть осторожным при использовании чужих формул по методу перевала. Сейчас будет сформулировано утверждение под названием ''Perron's formula''
\footnote{Формула (2.5) в файле "Метод перевала с остаточными членами". Сразу рассмотрим более частный случай, имеющий непосредственное влияние на нашу задачу. А именно возьмём перевальную точку второго порядка $m=2$, положим функцию рядом с экспонентой под интегралом $g(z)=1$, будем считать, что контур проходит через перевальную точку, а не имеет в ней начало или конец, как это приведено в книге.}. 
Это формула для нахождения интеграла через перевальную точку $z_0$ вдоль кривой наискорейшего спуска $\gamma$. 
\begin{equation}\label{steepest_decent}
\begin{aligned}
    \int_{\gamma} e^{\lambda f(z)} dz = e^{\lambda f(z_0)}\sum\limits_{n=0}^{\infty} \Gamma\left(n+\frac{1}{2}\right)\frac{c_{2n}}{\lambda^{n+\frac{1}{2}}} \\
    c_{2n} = \frac{1}{(2n)!}\left[\left(\frac{d}{dz}\right)^{2n}\left\lbrace\frac{(z-z_0)^2}{f(z)-f(z_0)}\right\rbrace^{n+\frac{1}{2}}\right]_{z=z_0}
\end{aligned}
\end{equation}
Если же использовать разложение функции $f$ в ряд Тейлора, то можно получить ''Campbell – Froman – Walles – Wojdylo formula''\footnote{формула (1.11) в книжке по методу перевала.}.
\begin{equation}\label{steepest_decent_taylor}
\begin{aligned}
    &f(z) = f(z_0) + \sum\limits_{p=0}^{\infty} a_p (z-z_0)^{p+2}\\
    &c_{2n} = \frac{1}{a_0^{n+\frac{1}{2}}}\sum\limits_{j=0}^{2n} C_{-n-\frac{1}{2}}^j\frac{1}{a_0^j}\hat{B}_{2n, j}\left(a_1, a_2, \dots, a_{2n-j+1}\right)
\end{aligned}
\end{equation}
\subsection*{Обобщённые числа Стирлинга}
Они упоминаются в английской вики на странице \href{https://en.wikipedia.org/wiki/Stirling_numbers_of_the_second_kind}{о числах Стирлинга}. Там же приводится ссылка на книгу Кометта, посвящённую комбинаторике. (см. papers)
\begin{equation}
\begin{aligned}
    \exp\left(u\left(\frac{t^r}{r!}+\frac{t^{r+1}}{\left(r+1\right)!}+\dots\right)\right) = \sum\limits_{n=(r+1)k, k = 0}^{\infty} S_{r}(n, k) u^k \frac{t^n}{n!}
\end{aligned}
\end{equation}

Для чисел Стирлинга есть рекуррентная формула всё в той же книжке ''Advanced combinatorics'':
\begin{equation}
\begin{aligned}
    S_r(n+1, k) = k S_r(n, k) + C_n^{r} S_r(n-r+1, k-1)
\end{aligned}
\end{equation}
\subsection*{Немного о полиномах Белла}
\begin{equation}
\begin{aligned}
    \exp\left(u \sum\limits_{j=1}^{\infty}x_j t^j\right) = \sum\limits_{n\geq k\geq 0} 
    \hat{B}_{n,k}\left(x_1, x_2, \dots, x_{n-k+1}\right) t^n \frac{u^k}{k!}
\end{aligned}
\end{equation}

Подставляя необходимые $x_j$ в нашем случае, получаем следующий ряд:

\begin{equation}
\begin{aligned}
    \sum\limits_{n\geq k\geq 0} 
    \hat{B}_{n, k}\left(\frac{1}{r!}, \frac{1}{(r+1)!}, \dots, \frac{1}{(n-k+r)!}\right) t^n \frac{u^k}{k!} &= 
    \exp\left(\frac{u}{t^{r-1}} \left(\frac{t^r}{r!} + \frac{t^{r+1}}{(r+1)!}+\dots \right)\right) \\ 
    &= \sum\limits_{n, k}^{\infty} S_{r}(n, k) \frac{u^k}{t^{(r-1)k}}\frac{t^n}{n!} \\
    &= \sum\limits_{n, k}^{\infty} S_{r}(n+(r-1)k, k) u^k\frac{t^n}{(n+(r-1)k)!}
\end{aligned}
\end{equation}


\begin{equation}
\begin{aligned}
    \hat{B}_{n, k}\left(\frac{1}{r!}, \frac{1}{(r+1)!}, \dots, \frac{1}{(n-k+r)!}\right) = 
    \frac{k!}{(n+(r-1)k)!}S_{r}(n+(r-1)k, k)
\end{aligned}
\end{equation}
\subsection*{Применение теории}
Как упоминается в приложении к диплому:
\begin{equation}
\begin{aligned}
    \sum\limits_{n=0}^{\infty} \frac{\alpha^n e^{i\gamma n^2}}{n!} 
    &= \frac{e^{\frac{i\pi}{4}}}{\sqrt{\pi\gamma}}
    \int\limits_{-\infty}^{\infty}e^{-i \frac{x^2}{\gamma} + \alpha e^{2ix}}dx
    &= \frac{e^{\frac{i\pi}{4}}}{\sqrt{\pi\gamma}}
    \int\limits_{-\infty}^{\infty}e^{\frac{1}{\gamma}\left(-i x^2 + \alpha \gamma e^{2ix}\right)}dx
\end{aligned}
\end{equation}
В таком виде очевидно, что в формуле \ref{steepest_decent} будут использоваться следующие замены: $\lambda\leadsto\frac{1}{\gamma}$, $f(x) \leadsto -i x^2 + \alpha \gamma e^{2ix} = -i x^2 - \frac{z}{2i} e^{2ix}$. (А так же вспомним обозначение из первой части $z=-2i\alpha\gamma$).
\begin{equation}
\begin{aligned}
    \frac{f(x) - f(z_k)}{(x-z_k)^2} 
    &=  \sum\limits_{p=0}^{\infty} \frac{f^{(p+2)}(z_k)}{(p+2)!}(x-z_k)^{p}\\
    &=  \underbrace{\left(-i - iz e^{2iz_k}\right)}_{a_0} - \sum\limits_{p=1}^{\infty} \underbrace{\frac{(2i)^{p+1}ze^{2iz_k}}{(p+2)!}}_{a_1, a_2, \dots}(x-z_k)^{p}
\end{aligned}
\end{equation}
Теперь мы готовы воспользоваться формулой \ref{steepest_decent_taylor} и выразить интеграл по перевальному контуру через $z_k$ (а так же используем формулу 5.6 из моего диплома):
\begin{equation}
\begin{aligned}
    &\int\limits_{\gamma_k}e^{\frac{1}{\gamma}\left(-i x^2 + \alpha \gamma e^{2ix}\right)}dx 
    = \exp\left(\frac{z_k(1-i z_k)}{\gamma}\right)\sum\limits_{n=0}^{\infty} \Gamma\left(n+\frac{1}{2}\right)c_{2n}\gamma^{n+\frac{1}{2}}\\
    &c_{2n} =
    \sum\limits_{j=0}^{2n} C_{-n-\frac{1}{2}}^j\frac{1}{\left(-i - iz e^{2iz_k}\right)^{n+j+\frac{1}{2}}}
    \hat{B}_{2n, j}\left(-\frac{(2i)^{2}ze^{2iz_k}}{3!}, -\frac{(2i)^{4}ze^{2iz_k}}{4!}, \dots, -\frac{(2i)^{2n-j+2}ze^{2iz_k}}{(2n-j+3)!}\right)
\end{aligned}
\end{equation}
Для упрощения последнего выражения нам понадобиться пара свойств полиномов Белла. А именно можно использовать их однородность и экспоненциальные полиномы Белла:
\begin{equation}
\begin{aligned}
    \hat{B}_{2n, j}(\zeta x_1, \zeta x_2, \dots, \zeta x_{2n-j+1}) &= \zeta^j \hat{B}_{2n, j}( x_1,  x_2, \dots,  x_{2n-j+1})\\
    \hat{B}_{2n, j}(\zeta x_1, \zeta^2 x_2, \dots, \zeta^{2n-j+1} x_{2n-j+1}) &= \zeta^{2n} \hat{B}_{2n, j}( x_1,  x_2, \dots,  x_{2n-j+1})
\end{aligned}
\end{equation}

Из этого следует:
\begin{equation}
\begin{aligned}
    \hat{B}_{2n, j}&\left(-\frac{(2i)^{2}ze^{2iz_k}}{3!}, -\frac{(2i)^{4}ze^{2iz_k}}{4!}, \dots, -\frac{(2i)^{2n-j+2}ze^{2iz_k}}{(2n-j+3)!}\right) = \\
    &= (-2iz e^{2iz_k})^j (2i)^{2n}\hat{B}_{2n, j}\left(\frac{1}{3!}, \frac{1}{4!}, \dots, \frac{1}{(2n-j+3)!}\right)
\end{aligned}
\end{equation}

Используя выше перечисленное\footnote{A также $\Gamma\left(\frac{1}{2}-n\right)= \frac{(2i)^{2n} n!}{(2n)!}\sqrt{\pi}$}, можно написать:

\begin{equation}
\begin{aligned}
    c_{2n} &=
    \sum\limits_{j=0}^{2n} C_{-n-\frac{1}{2}}^j
    \frac{(-2iz e^{2iz_k})^j (2i)^{2n}}{\left(-i - iz e^{2iz_k}\right)^{n+j+\frac{1}{2}}} 
    \frac{j!}{(2n+2j)!}S_{3}(2n+2j, j)\\
    &= \sum\limits_{j=0}^{2n} \frac{n!(2n+2j)!}{(n+j)!(2n)!j! (2i)^{2j}}
    \frac{(-2iz e^{2iz_k})^j (2i)^{2n}}{\left(-i - iz e^{2iz_k}\right)^{n+j+\frac{1}{2}}} 
    \frac{j!}{(2n+2j)!}S_{3}(2n+2j, j)\\
    &= \sum\limits_{j=0}^{2n} \frac{n!}{(n+j)!(2n)!}
    \frac{(-2iz e^{2iz_k})^j (2i)^{2n-2j}}{\left(-i - iz e^{2iz_k}\right)^{n+j+\frac{1}{2}}} S_{3}(2n+2j, j)
\end{aligned}
\end{equation}

\section*{Альтернативное переписывание}
Как можно было заметить, в полученных формулах много некрасивостей. Сейчас, когда мы уже знаем, какие выражения придётся ворочить, предлагается сделать следующие переобозначения:
\begin{equation}
\begin{aligned}
    A &\leadsto \alpha \\
    \Gamma &\leadsto \gamma \\
    Z = R e^{i\Phi}  = -2i A \Gamma&\leadsto -2i\alpha \gamma
\end{aligned}
\end{equation}

Мотивация следующая:
\begin{enumerate}
    \item Большие буквы обозначают неизменность, что важно в контексте множества сумм, парамтров и всего такого
    \item $A$ позволит не путать моё ошибочно выбранное обозначение с общепринятым
    \item $\Gamma$ совпадает с общепринятым обозначением
    \item Большая буква $Z$ обозначает неизменную комплексную величину\footnote{Надеюсь на благоразумие читателей, потому что автор против Z-движения в России}
\end{enumerate}

Так же для упрощения формул с методом перевала немного переписать подынтегральную функцию:
\begin{equation}
\begin{aligned}
    \sum\limits_{n=0}^{\infty} \frac{A^n e^{i\Gamma n^2}}{n!} 
    &= \frac{e^{\frac{i\pi}{4}}}{\sqrt{\pi\Gamma}}
    \int\limits_{-\infty}^{\infty}e^{-i \frac{z^2}{\Gamma} + A e^{2iz}}dz \\
    &= \frac{e^{\frac{i\pi}{4}}}{2\sqrt{\pi\Gamma}}
    \int\limits_{-\infty}^{\infty}e^{-i \frac{z^2}{4\Gamma} + A e^{iz}}dz \\
    &= \frac{e^{\frac{i\pi}{4}}}{2\sqrt{\pi\Gamma}}
    \int\limits_{-\infty}^{\infty}\exp\left(\frac{-i\frac{z^2}{2} + Z i e^{iz}}{2\Gamma}\right)dz 
\end{aligned}
\end{equation}

Сделаем следующее обозначение:
\begin{equation}
\begin{aligned}
    f(z) &=  \frac{z^2}{2i} + i Z e^{iz}
\end{aligned}
\end{equation}

Далее мы будем пользоваться методом перевала. Здесь нам нужно обосновать, что контур действительно можно деформировать ... Этому посвящена моя прога и анализ до этого момента, так что дописать это будет не трудно.

Теперь решения $f'(z_k) = 0$ можно просто обозначить в виде:
\begin{equation}
\begin{aligned}
    &e^{i z_k} = \frac{z_k}{iZ} \Rightarrow
    &z_k = i W_k(Z)
\end{aligned}
\end{equation}

По-моему, такие обозначения просто прекрасны. Вот, например, разложение вокруг перевальной точки:
\begin{equation}
\begin{aligned}
    f(z) = \underbrace{\frac{z_k^2}{2i} + z_k}_{f(z_k)} + \underbrace{\frac{-i-z_k}{2}}_{a_0} (z-z_k)^2 + 
    \sum_{n=1}^{\infty} \underbrace{\frac{-i^n z_k}{(n+2)!}}_{a_1, \: a_2, \dots} (z-z_k)^{n+2}
\end{aligned}
\end{equation}

Полагая $\lambda = \frac{1}{2\Gamma}$ можно написать\footnote{Надеюсь, что читатель не перепутает $\Gamma$ и $\Gamma$-функцию. Для удобства чтения после использования числа $\Gamma$ не будет скобочек.}:
\begin{equation}\label{const curve implement}
\begin{aligned}
    &\int\limits_{\gamma_k} \exp\left(\lambda f(z)\right) dz 
    = \exp\left(\lambda f(z_k)\right)\sum\limits_{n=0}^{\infty} \Gamma\left(n+\frac{1}{2}\right)\frac{c_{2n}}{\lambda^{n+\frac{1}{2}}}\\
    &c_{2n} =
    \sum\limits_{j=0}^{2n} C_{-n-\frac{1}{2}}^j\frac{1}{a_0^{n+j+\frac{1}{2}}}\hat{B}_{2n, j}\left(a_1, a_2, \dots, a_{2n-j+1}\right) \\
    &\;\;\;\;\; = \frac{1}{\left(\frac{-i-z_k}{2}\right)^{\frac{1}{2}}}
    \sum\limits_{j=0}^{2n} \frac{\Gamma\left(-n+\frac{1}{2}\right)}{j!\Gamma\left(-n-j+\frac{1}{2}\right)}
    \frac{1}{\left(\frac{-i-z_k}{2}\right)^{n+j}}
    (-1)^n (-z_k)^j\hat B_{2n, j}\left(\frac{1}{3!}, \frac{1}{4!}, \dots, \frac{1}{(2n-j+3)!}\right)\\
    &\;\;\;\;\; = \frac{1}{\left(\frac{-i-z_k}{2}\right)^{\frac{1}{2}}}
    \sum\limits_{j=0}^{2n} \frac{\frac{n!}{(2n)!}}{j!\frac{(-4)^j(n+j)!}{(2n+2j)!}}
    \frac{(-1)^n (-z_0)^j}{\left(\frac{-i-z_k}{2}\right)^{n+j}} \frac{j!}{(2n+2j)!}S_3(2n+2j, j)\\
    &\;\;\;\;\; = \frac{(-1)^n n!}{(2n)!}\frac{1}{\left(\frac{-i-z_k}{2}\right)^{\frac{1}{2}}}
    \sum\limits_{j=0}^{2n} \frac{1}{(-4)^j(n+j)!}
    \frac{ (-z_k)^j}{\left(\frac{-i-z_k}{2}\right)^{n+j}} S_3(2n+2j, j)
\end{aligned}
\end{equation}
\begin{equation}
\begin{aligned}
    \Gamma\left(n+\frac{1}{2}\right)c_{2n} &= \frac{\sqrt{2\pi}}{4^n \left(-i-z_k\right)^{\frac{1}{2}}}
    \sum\limits_{j=0}^{2n} \frac{1}{(-4)^j(n+j)!}
    \frac{ (-z_k)^j}{\left(\frac{-i-z_k}{2}\right)^{n+j}} S_3(2n+2j, j) \\
    &=\frac{\sqrt{2\pi}}{\left(-i-z_k\right)^{\frac{1}{2}}}
    \sum\limits_{j=0}^{2n}\left(-\frac{1}{2}\frac{z_k}{i+z_k}\right)^{n+j}\frac{S_3(2n+2j, j)}{(n+j)!}
\end{aligned}
\end{equation}
Резюмируя выше написанное одной формулой:
\begin{equation}
\begin{aligned}
    \sum\limits_{n=0}^{\infty}\left(\frac{iZ}{2\Gamma}\right)^n  \frac{e^{i\Gamma n^2}}{n!} &= e^{\frac{i\pi}{4}}
    \sum\limits_{k=0}^{-\sign\Gamma\cdot\infty} \frac{\exp\left(\frac{-i-i(z_k+i)^2}{4\Gamma}\right)}{(-i-z_k)^{\frac{1}{2}}}
    \sum\limits_{n=0}^{\infty}\left(2\Gamma\right)^n
    \sum\limits_{j=0}^{2n}\left(-\frac{1}{2}\frac{z_k}{i+z_k}\right)^{n+j}\frac{S_3(2n+2j, j)}{(n+j)!},
\end{aligned}
\end{equation}
где $z_k = i W_k(Z)$. От него так же можно избавиться и придти к следующую формулу:
\begin{equation}
\begin{aligned}
    \sum\limits_{n=0}^{\infty}\left(\frac{iZ}{2\Gamma}\right)^n  \frac{e^{i\Gamma n^2}}{n!} &= i
    \sum\limits_{k=0}^{-\sign\Gamma\cdot\infty} \frac{\exp\left(\frac{-i+i(1+W_k(Z))^2}{4\Gamma}\right)}{(1+W_k(Z))^{\frac{1}{2}}}
    \sum\limits_{n=0}^{\infty}\left(2\Gamma\right)^n
    \sum\limits_{j=0}^{2n}\left(-\frac{1}{2}\frac{W_k(Z)}{1+W_k(Z)}\right)^{n+j}\frac{S_3(2n+2j, j)}{(n+j)!},
\end{aligned}
\end{equation}

Поговорим немного о выборе ветви корня. Так как изначальный интеграл брался в пределах $\int\limits_{-\infty}^{\infty}$, то направление каждой из ветвей должно было выбираться из условия, чтоб $\arg\frac{dz}{ds}\in\left(-\frac{\pi}{2}, \frac{\pi}{2}\right)$, где $s$ --- натуральный параметр кривой постоянной фазы. Это значит:
\begin{equation}
\begin{aligned}
    \arg\left(-i-z_k\right)^{-\frac{1}{2}}\in\left(-\frac{\pi}{2}, \frac{\pi}{2}\right),
\end{aligned}
\end{equation}
что соответствует определению корня с разрезом в $z: \Real(z)\leq 0; \Imag(z)=0$. В последней же формуле используется разрез в $z: \arg(z) = -\frac{3\pi}{4}$, что соответствует $\Imag+\Real(\sqrt{1+W(Z)})>0$.


% \subsection*{Смелые идеи}
% В полученной сумме выделим лидирующий вклад при $n=0$. Он оценивается как $\frac{\sqrt{\pi}}{2\sqrt{\lambda}}\sum\limits_{k=0}^{-\infty}\exp\left(\lambda f(z_k)\right)$. Его можно было бы переписать в виде интеграла, а затем переписать полученное выражение с помощью метода перевала. Это получился бы метод перевала в квадрате! Покажу первые шаги:
% \begin{equation}
% \begin{aligned}
%     \sum\limits_{k=0}^{-\infty}\exp\left(\lambda f(z_k)\right) = 
%     \frac{1}{2\pi i}\int\limits_C \frac{e^{\lambda \tilde f(z)}}{-ize^{-iz}-Z} g(z)dz
% \end{aligned}
% \end{equation}

% Контур $C$ необходимо выбрать так, чтоб он обходил все $z_k$. Функции $\tilde f(z)$ и $g(z)$ необходимо выбрать из следующих соображений:
% \begin{equation}
% \begin{aligned}
%     &\tilde f (z_k) = f(z_k) \\
%     &\Res\limits_{z=z_k} \frac{e^{\lambda \tilde f(z)}}{-ize^{-iz}-Z} g(z) = e^{\lambda f(z_k)}
% \end{aligned}
% \end{equation}

% Или же, в предположении целости функции $g$:
% \begin{equation}
% \begin{aligned}
%     g(z_k) = \frac{d}{dz}\left(-ize^{-iz}-Z\right)_{z=z_k}  = \underbrace{-(i+z_k)e^{-iz_k}} = -(i+z_k)\frac{iZ}{z_k} = \underbrace{-i(1+Ze^{i z_k})e^{-iz_k}}
% \end{aligned}
% \end{equation}

% Любой из обведённых вариантов удобен для задании функции $g$ в зависимости от поведения модуля подынтегральной функции на комплексной плоскости. Очень важно, чтоб контур интегрирования деформировался удобным образом. Более простой вид для функции $\tilde f$ угадывается следующим образом:
% \begin{equation}
% \begin{aligned}
%     \tilde f(z) = \frac{z^2}{2i} + z = \frac{i}{2} + \frac{(z+i)^2}{2i}
% \end{aligned}
% \end{equation}

% Можно было бы выбрать и другие формы, однако в таком виде очевидным образом находится единственная перевальная точка:
% \begin{equation}
% \begin{aligned}
%     &\tilde z = -i\\
%     &\tilde f(\tilde z) = \frac{i}{2}
% \end{aligned}
% \end{equation}
    

\section*{Какое $k$ вносит основной вклад}
В полученном выражении стоит сумма по $k$ с лидирующими вкладами вида $\exp\left(\frac{i+(z_k+i)^2}{4\Gamma}\right)$. Стоит заметить, что, например, при $\Gamma = 10^{-6}$ , а значит незначительная разница на $0.01$ в $\Real\left((z_k+i)^2\right)$ при различных $k$ приведёт к разнице слагаемых в $\sim e^{-10^4}$, что значит, что меньшим из слагаемых можно пренебречь, не сильно потеряв в точности. Найдём, при каком $k$ вносится основной вклад. 
\begin{equation}
\begin{aligned}
    \Real\left(f(z_k)\right) = \frac{1}{2}\Real\left(-i(z_k+i)^2\right) = \frac{1}{2}\Imag\left((z_k+i)^2\right) 
    &= \Real\left(z_k\right)\left(1+\Imag(z_k)\right)\\
    &= -\Imag(W_k(Z))\left(1+\Real(W_k(Z))\right)
\end{aligned}
\end{equation}
Применим разложение, с которого начинается данный документ, огрубив его до суммирования ($R, |k| \rightarrow \infty$):
\begin{equation}\label{W_k_approx}
\begin{aligned}
    &W_k\left(Z=R e^{i\Phi}\right) = \log R + i\Phi + 2\pi i k - \log\left(\log R + i\Phi+ 2\pi i k\right) + O\left(\frac{\log\left(\log R + i\Phi+ 2\pi i k\right)}{\log R + i\Phi + 2\pi i k}\right)\\
    &= \log R + i\Phi + 2\pi i k - \log\left(\log R + 2\pi i k\right) + \underbrace{\log\left(1+\frac{i\Phi}{\log R +  2\pi i k}\right)}_{\sim \frac{1}{\log R  + 2\pi i k} \approx 0} + O\left(\frac{\log\left(\log R + 2\pi i k\right)}{\log R  + 2\pi i k}\right)\\
    &= \log R - \frac{1}{2}\log\left(\log^2 R + (2\pi k)^2\right) + 
    i\left(\Phi + 2\pi k - \arctan\left(\frac{2\pi k}{\log R}\right)\right) 
    + O\left(\frac{\log\left(\log R + 2\pi i k\right)}{\log R  + 2\pi i k}\right)
\end{aligned}
\end{equation}
% Из этого, если ненадолго продолжить данное выражение на область действительных $k$:
% \begin{equation}
% \begin{aligned}
%     &\frac{1}{2}\frac{d}{dk} \Real\left((z_k+i)^2\right) = \Imag\left(W_k(Z)\right)\frac{d \Imag(W_k(Z))}{dk} - 
%     \left(\Real(W_k(Z))+1\right) \frac{d \Real(W_k(Z))}{dk} = 0
% \end{aligned}
% \end{equation}

Воспользуемся сначала самым простым и тупым приближением: $W_k(Z) \approx \log R + i\Phi + 2\pi i k$
\begin{equation}
\begin{aligned}
    \Real\left(f(z_k)\right) \approx -(2\pi k+\Phi)(1+\log R)
\end{aligned}
\end{equation}
Получаем, что в первом приближении $k\approx \infty$.
Воспользуемся следующим порядком приближения без учёта суммы:
\begin{equation}
\begin{aligned}
    \Real\left(f(z_k)\right) &\approx
    -\left(\Phi + 2\pi k - \arctan\left(\frac{2\pi k}{\log R}\right)\right)\left(1+\log R - \frac{1}{2}\log\left(\log^2 R + (2\pi k)^2\right)\right)\\
    &\approx -\left(\Phi + 2\pi k - \frac{\pi}{2}\sign k\right)\left(1+\log R - \log|2\pi k|\right),
\end{aligned}
\end{equation}
где мы пренебрегли всеми слагаемыми, которые при раскрытии скобок дадут вклад $O\left(\frac{\log^2 R}{R}\right)$. Попробуем оптимизировать данное выражение так, как если бы $k\in\mathbb{R}$:
\begin{equation}
\begin{aligned}
    \frac{d}{dk}\Real\left(f(z_k)\right) = 0\\
\end{aligned}
\end{equation}

\begin{equation}
\begin{aligned}
    2\pi\left(1+\log R - \log|2\pi k|\right) &= \frac{1}{k}\left(\Phi + 2\pi k - \frac{\pi}{2}\sign k\right)\\
    \frac{2\pi |k|}{R}\log\left(\frac{2\pi |k|}{R}\right) &= \frac{\left|\Phi - \frac{\pi}{2}\sign k\right|}{R}
\end{aligned}
\end{equation}
Введём временно обозначение $e^\varkappa = \frac{2\pi |k|}{R} > 0$:
\begin{equation}
\begin{aligned}
    \varkappa e^{\varkappa} = \frac{\left|\Phi - \frac{\pi}{2}\sign k\right|}{R}\approx 0\\
    \varkappa = W_0\left(\frac{\left|\Phi - \frac{\pi}{2}\sign k\right|}{R}\right)
\end{aligned}
\end{equation}
Для нулевой функции Ламберта у нас так же есть небольшое разложение $W_0(x)\approx x$. Применим его и подставим $\varkappa$:
\begin{equation}
\begin{aligned}
    |k|= \left[ \frac{1}{2\pi}\left(R + \left|\Phi - \frac{\pi}{2}\sign k\right|\right)\right]
\end{aligned}
\end{equation}
Такое приближённое решение почти всегда даёт наилучшее приближение, потому что при учёте высших порядков вклад будет $\ll 1$ при больших $R$. Значения функции Ламберта в окрестности малых $R$ требует уже численных расчётов. Далее мы будем обозначать это оптимальное $k$ через $\overline k$, подразумевая его зависимость от $Z$ и $\sign \Gamma$.

\subsection*{Какой вклад вносит слагаемое под номером $\overline{k}$}
Как бы нам не хотелось убрать знак округления (квадратные скобки в формуле выше), оно неизбежно. Введём следующее обозначение:
\begin{equation}
\begin{aligned}
    \varkappa &= 2\pi \overline{k} - R\sign k\sim 1\\
    2\pi |\overline{k}| &= R + \varkappa \sign k
\end{aligned}
\end{equation}

\subsubsection*{Максимум по $\Phi$}
Будем рассматривать только показатель экспоненты основной ветви, считая, что она одна на всём промежутке $(-\pi, \pi)$.
Нам понадобиться формула для производной функции Ламберта:
\begin{equation}
\begin{aligned}
    \frac{dW}{dZ} = \frac{1}{Z}\frac{W}{1+W},
\end{aligned}
\end{equation}
где подразумевается тот же аргумент $Z$.
Для поиска точки максимума по $\Phi$ сделаем следующее:
\begin{equation}
\begin{aligned}
    \frac{d}{d\Phi}\Real\left(f(z_k)\right) &= 
    \frac{1}{2}\Imag\left(\frac{d}{d\Phi} (iW_k(Z) + i)^2\right) \\
    &= -\Imag\left((W_k(Z) + 1)\frac{dW_k(Z)}{d\Phi}\right)\\
    &= -\Imag\left((W_k(Z) + 1) \underbrace{\frac{d\log Z}{d\Phi}}_{=i} \frac{W_k(Z)}{1+W_k(Z)} \right)\\
    &= -\Imag\left(i W_k(Z)\right) = -\Real(W_k(Z)) = 0
\end{aligned}
\end{equation}

Воспользуемся определением $W$-функции Ламберта и возьмём модуль от обоих частей:
\begin{equation}
\begin{aligned}
    i\Imag W_k(Z)e^{i\Imag W_k(Z)} = R e^{i\Phi} \Rightarrow \left|\Imag W_k(Z)\right| = R
\end{aligned}
\end{equation}
Тут у нас встаёт небольшой вопрос о выборе знака. Для этого посмотрим на значение $\Real f(z_k)$ и учтём, что $\sign k = -\sign \Gamma$:
\begin{equation}
\begin{aligned}
    \sign\left(\frac{f(z_k)}{\Gamma}\right) = \sign k \sign\left(\Imag W_k(Z)\right) = 1 \Rightarrow \\\Imag W_k(Z) = R\sign k \Rightarrow \\
    \left(\frac{f(z_k)}{2\Gamma}\right)_{\Phi = \Phi_{max}} = \frac{R}{2|\Gamma|}
\end{aligned}
\end{equation}



А это обстоятельство очень полезно для определения угла $\Phi$:
\begin{equation}
\begin{aligned}
    &i R \sign k  e^{i R \sign k } = R e^{i\Phi} = R e^{i\frac{\pi}{2}\sign k + i R \sign k} \Rightarrow \\ 
    &\Phi = \left(R+\frac{\pi}{2}\right)\sign k \;\;\; \mod 2\pi
\end{aligned}
\end{equation}

Важным здесь является то, что мы получили этот результат не использовав ни одного приближения.
\subsection*{Граница выбора $z_k$}
На комплексной плоскости для любого $Z$ можно указать, каков оптимальный выбор $\bar{k}$. Этот выбор полностью определяется максимумом $\Real\left(f(z_k)\right) = -\Imag(W_k(Z))\left(1+\Real(W_k(Z))\right)$. 
Данная функция дискретна по $k$ и не имеет простого описания при малых $Z$. На границе:

\begin{equation}
\begin{aligned}
    \Real\left(f(z_k)\right) &= \Real\left(f(z_{k+\sign k})\right) \\
\end{aligned}
\end{equation}

Давайте посмотрим на асимптотику $|Z|\gg 1$. В этом случае мы снова применяем асимптотическое выражение\footnote{Чудовищные выкладки можно найти в моём вольфраме} для поиска пограничного $Z$ при $2\pi k= R\sign k + \varkappa$, где $\varkappa = O(1)$:
\begin{equation}
\begin{aligned}
    &L_1 = \log R + i \Phi + i\left(\underbrace{R\sign k + \varkappa}_{2\pi k}\right) = O(R)\\
    &L_2 = \log(L_1) = O(\log(R))\\
    &W_k(Z) = L_1 - L_2 + \frac{L_2}{L_1} + \frac{L_2(-2+L_2)}{2L_1^2} + O\left(\frac{\log^3 R}{R^3}\right) = \dots\\
    &\Real(f(z_k)) =  -R\sign k + \frac{\left(\Phi + \varkappa - \frac{\pi}{2}\sign k\right)^2}{2R} +
     O\left(\frac{\log^3 R}{R^2}\right)
\end{aligned}
\end{equation}
Замена $k$ на $k+\sign k$ в данных обозначениях это замена $\varkappa$ на $\varkappa + 2\pi\sign k$. На границе:
\begin{equation}
\begin{aligned}
    &-\frac{\pi\sign k}{R} + \frac{2}{R}
    \left(\varkappa + \Phi + \pi\sign k\right) = O\left(\frac{\log^3 R}{R^2}\right)\\
    &\varkappa + \Phi + \frac{\pi}{2}\sign k= O\left(\frac{\log^3 R}{R}\right)
\end{aligned}
\end{equation}
Или если интересоваться только значением $\Phi$:
\begin{equation}
\begin{aligned}
    \Phi \approx \left(R-\frac{\pi}{2}\right)\sign k \;\;\; \mod 2\pi
\end{aligned}
\end{equation}

\subsection*{И всё же об асимптотике}

Окей, мы точно знаем, какое надо выбирать $\bar k$ для каждого $Z$. Мы знаем, что вклад от остальных слагаемых крайне мал. Ещё, если немного углубиться в вычисления чисел Стирлинга 2 рода 3 типа, то можно заметить, что $S(2, 2+2k)=0$, а значит первый необнуляющийся член в разложении метода перевала вносит коррекцию по относительной величине порядка $\Gamma^2$. Временно пренебрежём и этим вкладом:

\begin{equation}
\begin{aligned}
    \sum\limits_{n=0}^{\infty}\left(\frac{iZ}{2\Gamma}\right)^n  \frac{e^{i\Gamma n^2}}{n!} \approx 
    e^{\frac{i\pi}{4}} \frac{\exp\left(\frac{-i-i(z_{\bar k}+i)^2}{4\Gamma}\right)}{(-i-z_{\bar k})^{\frac{1}{2}}}
\end{aligned}
\end{equation}

Эту же формулу можно ещё упростить, если считать $|Z|\gg 1$. Ещё проще она станет, если мы будем интересоваться лишь модулем полученного выражения. Чудовищные выкладки опять же в вольфраме. Отмечу лишь, что при рассчёте $W_k(Z)$ необходимо удерживать члены $O\left(\frac{1}{R^2}\right)$, чтоб получить значение $f(z_k)$ с точностью $O\left(\frac{1}{R}\right)$, так как $W_k(Z) = O(R)$.

\begin{equation}
\begin{aligned}
    \left|\sum\limits_{n=0}^{\infty}\left(\frac{iZ}{2\Gamma}\right)^n  \frac{e^{i\Gamma n^2}}{n!} \right|\approx 
    \frac{\exp\left(\frac{R}{2|\Gamma|} - \frac{\delta^2}{4R|\Gamma|}+\dots\right)}
    {\sqrt{R}}\exp\left(-\frac{\delta\sign k}{2R} + \frac{\delta^2}{4R^2}\right)\left(1-\frac{1}{4 R^2} + \frac{5}{32 R^4}+\dots\right),
\end{aligned}
\end{equation}
где мы ввели обозначение $\Phi + \varkappa - \frac{\pi}{2}\sign k  = \delta$. Эти из поправки к максимуму по $\Phi$ вымученные $\frac{5}{32}$ очень похожи на еле различимую $\frac{1}{6}$ в моём дипломе. А линейная добавка в экспоненту $\sim\frac{\delta}{R}$ так же отлично согласуется с численными расчётами, потому что в действительности максимум наблюдался не при $\delta=0$:
\begin{equation}
\begin{aligned}
    \left|\sum\limits_{n=0}^{\infty}\left(\frac{iZ}{2\Gamma}\right)^n  \frac{e^{i\Gamma n^2}}{n!} \right|\approx 
    \frac{1}{\sqrt{R}}\left(1-\frac{1}{4 R^2} + \frac{5}{32 R^4}\right)
    \exp\left(\frac{R}{2|\Gamma|} - \frac{\left(\delta -\Gamma\right)^2 - \Gamma^2}{4R|\Gamma|}\right)
\end{aligned}
\end{equation}
И это просто превосходно сходится с моим дипломом!!!
\section*{О кривых постоянной фазы}

Ключевым в данном рассказе является формула \ref{const curve implement}. Для её применения необходимо доказать, что мы действительно можем деформировать контур так, чтоб при $\Gamma < 0$ необходимо было бы учитывать все кривые постоянной фазы с $k\geq 0$. Это мы сделаем в несколько этапов:
\begin{enumerate}
    \item Покажем, что кривая постоянной фазы, проходящая через $z_0$ имеет асимптотическое поведение вида слева $x+y=const.$. 
    \item Покажем, что кривые постоянной фазы при $k>0$ имеют асимптотику вниз в области, где модуль стремится к нулю очень быстро
    \item Укажем деформацию контура на промежутке $(-R, R)$. При этом вылезут остаточные члены от вкладов по вертикальным линиям
    \item Заметим, что вклад от вертикальных линий асимптотически стремится к 0.
\end{enumerate}
Уравнение на кривую постоянной фазы $z(s)=x(s)+i y(s)$, $\left|\frac{dz}{ds}\right| = 1$:
\begin{eqw}
    \Imag f(z) = \Imag f(z_k) \\
    0=\Imag \frac{df}{dz}\frac{dz}{ds} = -\Imag\left((iz+Ze^{iz})\frac{dz}{ds}\right) = u\frac{dx}{ds} + v\frac{dy}{ds}\\
    \left\{
    \begin{aligned}
        \frac{dx}{ds} =\pm \frac{v}{\sqrt{u^2+v^2}}\\
        \frac{dy}{ds} =\mp \frac{u}{\sqrt{u^2+v^2}}
    \end{aligned}
    \right.
\end{eqw}
Пойдём последовательно.
\subsection*{Кривая постоянной фазы при $k=0$}

\end{document}