\documentclass[a4paper, 12pt]{article}

%Русский язык
\renewcommand{\familydefault}{\sfdefault}%шрифт
\usepackage[T2A]{fontenc} %кодировка
\usepackage[utf8]{inputenc} %кодировка исходного кода
\usepackage[english,russian]{babel} %локализация и переносы
%отступы 
\usepackage[left=2cm,right=2cm,top=2cm,bottom=3cm,bindingoffset=0cm]{geometry}
\usepackage{indentfirst}
%Вставка картинок
\usepackage{graphicx}
\graphicspath{}
\DeclareGraphicsExtensions{.pdf,.png,.jpg, .jpeg}

%Таблицы
\usepackage[table,xcdraw]{xcolor}
\usepackage{booktabs}

% Cсылки
\usepackage{hyperref}
%Математика
\usepackage{amsmath, amsfonts, amssymb, amsthm, mathtools }

%Заголовок
\author{Нугманов Булат}

\begin{document}
\section*{Разложение $z_k$}
\subsection*{Разложение функции Ламберта из статьи}
Следующая формула взята из статьи "On the Lambert W Function"(DOI:10.1007/BF02124750), формула (4.20):
\begin{equation}
\begin{aligned}
    W_k(z) = \log z + 2\pi i k - \log\left(\log z + 2\pi i k\right) + \sum\limits_{k=0}^{\infty}\sum\limits_{m=1}^{\infty} c_{km}\log^m\left(\log z + 2\pi i k\right)\left(\log z + 2\pi i k \right)^{-k-m}
\end{aligned}
\end{equation}

Для того, чтоб ветви функции Ламберта совпадали с общепринятыми, ветви $\log z$ необходимо так же брать привычными --- с разрезом на отрицательных числах и нулевой мнимой частью при положительных $z$. Коэффициенты $c_{km}$ определены в статье после формулы (4.18):
\begin{equation}
\begin{aligned}
    c_{km} = \frac{(-1)^k}{m!} c(k+m, k+1)
\end{aligned}
\end{equation}

$c(k+m, k+1)$ --- это беззнаковые числа Стирлинга первого рода. В вольфраме они обозначаются как "Abs@StirlingS1[k+m, k+1]".

В нашей же задаче, требуется определить $z_k = \frac{i}{2}W_k(-2i\alpha\gamma)$. Обозначая $z=-2i\alpha\gamma$, $k+m=n$, получаем:
\begin{equation}
\begin{aligned}
    -2iz_k &= W_k(z) \\
    &= \log z + 2\pi i k - \log\left(\log z + 2\pi i k\right) + \dots \\
    &\dots + \sum\limits_{n=1}^{\infty}\sum\limits_{m=1}^{n} 
    \frac{(-1)^{n-m}}{m!} c(n, n-m+1)\frac{\log^{m}\left(\log z + 2\pi i k\right)}{\left(\log z + 2\pi i k\right)^n}
\end{aligned}
\end{equation}
В такой форме наглядно видно разложение по малости остаточных членов. В дальнейшем будет видно, что большим параметром при разложении здесь является номер функции Ламберта --- $k$. Ещё можно использовать знаковые числа Стирлинга ($s(n, k) = (-1)^{n-k}c(n, k) \Rightarrow (-1)^{n-m} c(n, n-m+1) = (-1)^{n+1} s(n, n-m+1)$), однако в этом нет пока необходимости.

\section*{Метод перевала с остаточными членами}
\subsection*{Общая теория метода перевала второго порядка}
В википедии следующая штука имеет название "Campbell–Froman–Walles–Wojdylo formula". Если нужно куда-то сослаться, то подойдёт файл "Метод перевала с остаточными членами". Там это формула (1.11). (Для нас удобнее наблюдать за последней строчкой. Для случая второго порядка перевальной точки надо положить $\alpha=2$. Для случая отсутствующей при экспоненте функции под интегралом надо положить $b_k = \delta_{k0}, \: \beta = 1$.) Пример для случая перевальной точки второго порядка на контуре интегрирования можно найти в формулах (1.17 - 1.19). (Там пример с асимптотикой гамма-фукнции.) Перепишем то же самое для случая перевальной точки второго порядка $z_0$:

\begin{equation}\label{steepest_decent}
\begin{aligned}
    &\int e^{\lambda f(z)} dz = \frac{e^{\lambda f(z_0)+i\chi}}{\sqrt{|a_0|\lambda}}\sum\limits_{n=0}^{\infty} \Gamma\left(n+\frac{1}{2}\right)\frac{c_{2n}}{\lambda^n} \\ 
    & \Gamma\left(n+\frac{1}{2}\right) c_{2n}= \frac{1}{a_0^n}\sum\limits_{j=0}^{2n}\frac{(-1)^j}{j! a_0^j} \Gamma\left(n+j+\frac{1}{2}\right) \hat{B}_{2n, j}\left(a_1, a_2, \dots, a_{2n-j+1}\right) \\
    & f(z) = f(z_0) + \sum\limits_{m=0}^{\infty} a_k (z-z_0)^{m+2},
\end{aligned}
\end{equation}
где $\chi$ --- угол, характеризующий направление кривой постоянной фазы в окрестности точки перевала, а $\hat{B}_{2n, j}\left(\dots\right)$ --- обыкновенные полиномы Белла. В большинстве литературы рассматриваются исключительно экспоненциальные полиномы Белла, которые в обозначениях отличаются лишь шляпкой.
\subsection*{Обобщённые числа Стирлинга}
Они упоминаются в английской вики на странице \href{https://en.wikipedia.org/wiki/Stirling_numbers_of_the_second_kind}{о числах Стирлинга}. Там же приводится ссылка на книгу Кометта, посвящённую комбинаторике. (см. papers)
\subsection*{Применение теории}
Как упоминается в приложении к диплому:
\begin{equation}\label{steepest_decent}
\begin{aligned}
    \sum\limits_{n=0}^{\infty} \frac{\alpha^n e^{i\gamma n^2}}{n!} 
    &= \frac{e^{\frac{i\pi}{4}}}{\sqrt{\pi\gamma}}
    \int\limits_{-\infty}^{\infty}e^{-i \frac{x^2}{\gamma} + \alpha e^{2ix}}dx
    &= \frac{e^{\frac{i\pi}{4}}}{\sqrt{\pi\gamma}}
    \int\limits_{-\infty}^{\infty}e^{\frac{1}{\gamma}\left(-i x^2 + \alpha \gamma e^{2ix}\right)}dx
\end{aligned}
\end{equation}
В таком виде очевидно, что в формуле \ref{steepest_decent} будут использоваться следующие замены: $\lambda\leadsto\frac{1}{\gamma}$, $f(x) = -i x^2 + \alpha \gamma e^{2ix} = -i x^2 + \frac{iz}{2} e^{2ix}$. (А так же вспомним обозначение из первой части $z=-2i\alpha\gamma$).
\end{document}