\documentclass[a4paper, 12pt]{article}

%Русский язык
\renewcommand{\familydefault}{\sfdefault}%шрифт
\usepackage[T2A]{fontenc} %кодировка
\usepackage[utf8]{inputenc} %кодировка исходного кода
\usepackage[english,russian]{babel} %локализация и переносы
%отступы 
\usepackage[left=2cm,right=2cm,top=2cm,bottom=3cm,bindingoffset=0cm]{geometry}
\usepackage{indentfirst}
%Вставка картинок
\usepackage{graphicx}
\graphicspath{}
\DeclareGraphicsExtensions{.pdf,.png,.jpg, .jpeg}

%Таблицы
\usepackage[table,xcdraw]{xcolor}
\usepackage{booktabs}

% Cсылки
\usepackage{hyperref}
%Математика
\usepackage{amsmath, amsfonts, amssymb, amsthm, mathtools }
\DeclareMathOperator*{\Res}{Res}
%Заголовок
\author{Нугманов Булат}

\begin{document}
\section*{Разложение $z_k$}
\subsection*{Разложение функции Ламберта из статьи}
Следующая формула взята из статьи "On the Lambert W Function"(DOI:10.1007/BF02124750), формула (4.20):
\begin{equation}
\begin{aligned}
    W_k(z) = \log z + 2\pi i k - \log\left(\log z + 2\pi i k\right) + \sum\limits_{k=0}^{\infty}\sum\limits_{m=1}^{\infty} c_{km}\log^m\left(\log z + 2\pi i k\right)\left(\log z + 2\pi i k \right)^{-k-m}
\end{aligned}
\end{equation}

Для того, чтоб ветви функции Ламберта совпадали с общепринятыми, ветви $\log z$ необходимо так же брать привычными --- с разрезом на отрицательных числах и нулевой мнимой частью при положительных $z$. Коэффициенты $c_{km}$ определены в статье после формулы (4.18):
\begin{equation}
\begin{aligned}
    c_{km} = \frac{(-1)^k}{m!} c(k+m, k+1)
\end{aligned}
\end{equation}

$c(k+m, k+1)$ --- это беззнаковые числа Стирлинга первого рода. В вольфраме они обозначаются как "Abs@StirlingS1[k+m, k+1]".

В нашей же задаче, требуется определить $z_k = \frac{i}{2}W_k(-2i\alpha\gamma)$. Обозначая $z=-2i\alpha\gamma$, $k+m=n$, получаем:
\begin{equation}
\begin{aligned}
    -2iz_k &= W_k(z) \\
    &= \log z + 2\pi i k - \log\left(\log z + 2\pi i k\right) + \dots \\
    &\dots + \sum\limits_{n=1}^{\infty}\sum\limits_{m=1}^{n} 
    \frac{(-1)^{n-m}}{m!} c(n, n-m+1)\frac{\log^{m}\left(\log z + 2\pi i k\right)}{\left(\log z + 2\pi i k\right)^n}
\end{aligned}
\end{equation}
В такой форме наглядно видно разложение по малости остаточных членов. В дальнейшем будет видно, что большим параметром при разложении здесь является номер функции Ламберта --- $k$. Ещё можно использовать знаковые числа Стирлинга ($s(n, k) = (-1)^{n-k}c(n, k) \Rightarrow (-1)^{n-m} c(n, n-m+1) = (-1)^{n+1} s(n, n-m+1)$), однако в этом нет пока необходимости.

\section*{Метод перевала с остаточными членами}
\subsection*{Общая теория метода перевала}
Следует быть осторожным при использовании чужих формул по методу перевала. Сейчас будет сформулировано утверждение под названием ''Perron's formula''
\footnote{Формула (2.5) в файле "Метод перевала с остаточными членами". Сразу рассмотрим более частный случай, имеющий непосредственное влияние на нашу задачу. А именно возьмём перевальную точку второго порядка $m=2$, положим функцию рядом с экспонентой под интегралом $g(z)=1$, будем считать, что контур проходит через перевальную точку, а не имеет в ней начало или конец, как это приведено в книге.}. 
Это формула для нахождения интеграла через перевальную точку $z_0$ вдоль кривой наискорейшего спуска $\gamma$. 
\begin{equation}\label{steepest_decent}
\begin{aligned}
    \int_{\gamma} e^{\lambda f(z)} dz = e^{\lambda f(z_0)}\sum\limits_{n=0}^{\infty} \Gamma\left(n+\frac{1}{2}\right)\frac{c_{2n}}{\lambda^{n+\frac{1}{2}}} \\
    c_{2n} = \frac{1}{(2n)!}\left[\left(\frac{d}{dz}\right)^{2n}\left\lbrace\frac{(z-z_0)^2}{f(z)-f(z_0)}\right\rbrace^{n+\frac{1}{2}}\right]_{z=z_0}
\end{aligned}
\end{equation}
Если же использовать разложение функции $f$ в ряд Тейлора, то можно получить ''Campbell – Froman – Walles – Wojdylo formula''\footnote{формула (1.11) в книжке по методу перевала.}.
\begin{equation}\label{steepest_decent_taylor}
\begin{aligned}
    &f(z) = f(z_0) + \sum\limits_{p=0}^{\infty} a_p (z-z_0)^{p+2}\\
    &c_{2n} = \frac{1}{a_0^{n+\frac{1}{2}}}\sum\limits_{j=0}^{2n} C_{-n-\frac{1}{2}}^j\frac{1}{a_0^j}\hat{B}_{2n, j}\left(a_1, a_2, \dots, a_{2n-j+1}\right)
\end{aligned}
\end{equation}
\subsection*{Обобщённые числа Стирлинга}
Они упоминаются в английской вики на странице \href{https://en.wikipedia.org/wiki/Stirling_numbers_of_the_second_kind}{о числах Стирлинга}. Там же приводится ссылка на книгу Кометта, посвящённую комбинаторике. (см. papers)
\begin{equation}
\begin{aligned}
    \exp\left(u\left(\frac{t^r}{r!}+\frac{t^{r+1}}{\left(r+1\right)!}+\dots\right)\right) = \sum\limits_{n=(r+1)k, k = 0}^{\infty} S_{r}(n, k) u^k \frac{t^n}{n!}
\end{aligned}
\end{equation}

Для чисел Стирлинга есть рекуррентная формула всё в той же книжке ''Advanced combinatorics'':
\begin{equation}
\begin{aligned}
    S_r(n+1, k) = k S_r(n, k) + C_n^{r} S_r(n-r+1, k-1)
\end{aligned}
\end{equation}
\subsection*{Немного о полиномах Белла}
\begin{equation}
\begin{aligned}
    \exp\left(u \sum\limits_{j=1}^{\infty}x_j t^j\right) = \sum\limits_{n\geq k\geq 0} 
    \hat{B}_{n,k}\left(x_1, x_2, \dots, x_{n-k+1}\right) t^n \frac{u^k}{k!}
\end{aligned}
\end{equation}

Подставляя необходимые $x_j$ в нашем случае, получаем следующий ряд:

\begin{equation}
\begin{aligned}
    \sum\limits_{n\geq k\geq 0} 
    \hat{B}_{n, k}\left(\frac{1}{r!}, \frac{1}{(r+1)!}, \dots, \frac{1}{(n-k+r)!}\right) t^n \frac{u^k}{k!} &= 
    \exp\left(\frac{u}{t^{r-1}} \left(\frac{t^r}{r!} + \frac{t^{r+1}}{(r+1)!}+\dots \right)\right) \\ 
    &= \sum\limits_{n, k}^{\infty} S_{r}(n, k) \frac{u^k}{t^{(r-1)k}}\frac{t^n}{n!} \\
    &= \sum\limits_{n, k}^{\infty} S_{r}(n+(r-1)k, k) u^k\frac{t^n}{(n+(r-1)k)!}
\end{aligned}
\end{equation}


\begin{equation}
\begin{aligned}
    \hat{B}_{n, k}\left(\frac{1}{r!}, \frac{1}{(r+1)!}, \dots, \frac{1}{(n-k+r)!}\right) = 
    \frac{k!}{(n+(r-1)k)!}S_{r}(n+(r-1)k, k)
\end{aligned}
\end{equation}
\subsection*{Применение теории}
Как упоминается в приложении к диплому:
\begin{equation}
\begin{aligned}
    \sum\limits_{n=0}^{\infty} \frac{\alpha^n e^{i\gamma n^2}}{n!} 
    &= \frac{e^{\frac{i\pi}{4}}}{\sqrt{\pi\gamma}}
    \int\limits_{-\infty}^{\infty}e^{-i \frac{x^2}{\gamma} + \alpha e^{2ix}}dx
    &= \frac{e^{\frac{i\pi}{4}}}{\sqrt{\pi\gamma}}
    \int\limits_{-\infty}^{\infty}e^{\frac{1}{\gamma}\left(-i x^2 + \alpha \gamma e^{2ix}\right)}dx
\end{aligned}
\end{equation}
В таком виде очевидно, что в формуле \ref{steepest_decent} будут использоваться следующие замены: $\lambda\leadsto\frac{1}{\gamma}$, $f(x) \leadsto -i x^2 + \alpha \gamma e^{2ix} = -i x^2 - \frac{z}{2i} e^{2ix}$. (А так же вспомним обозначение из первой части $z=-2i\alpha\gamma$).
\begin{equation}
\begin{aligned}
    \frac{f(x) - f(z_k)}{(x-z_k)^2} 
    &=  \sum\limits_{p=0}^{\infty} \frac{f^{(p+2)}(z_k)}{(p+2)!}(x-z_k)^{p}\\
    &=  \underbrace{\left(-i - iz e^{2iz_k}\right)}_{a_0} - \sum\limits_{p=1}^{\infty} \underbrace{\frac{(2i)^{p+1}ze^{2iz_k}}{(p+2)!}}_{a_1, a_2, \dots}(x-z_k)^{p}
\end{aligned}
\end{equation}
Теперь мы готовы воспользоваться формулой \ref{steepest_decent_taylor} и выразить интеграл по перевальному контуру через $z_k$ (а так же используем формулу 5.6 из моего диплома):
\begin{equation}
\begin{aligned}
    &\int\limits_{\gamma_k}e^{\frac{1}{\gamma}\left(-i x^2 + \alpha \gamma e^{2ix}\right)}dx 
    = \exp\left(\frac{z_k(1-i z_k)}{\gamma}\right)\sum\limits_{n=0}^{\infty} \Gamma\left(n+\frac{1}{2}\right)c_{2n}\gamma^{n+\frac{1}{2}}\\
    &c_{2n} =
    \sum\limits_{j=0}^{2n} C_{-n-\frac{1}{2}}^j\frac{1}{\left(-i - iz e^{2iz_k}\right)^{n+j+\frac{1}{2}}}
    \hat{B}_{2n, j}\left(-\frac{(2i)^{2}ze^{2iz_k}}{3!}, -\frac{(2i)^{4}ze^{2iz_k}}{4!}, \dots, -\frac{(2i)^{2n-j+2}ze^{2iz_k}}{(2n-j+3)!}\right)
\end{aligned}
\end{equation}
Для упрощения последнего выражения нам понадобиться пара свойств полиномов Белла. А именно можно использовать их однородность и экспоненциальные полиномы Белла:
\begin{equation}
\begin{aligned}
    \hat{B}_{2n, j}(\zeta x_1, \zeta x_2, \dots, \zeta x_{2n-j+1}) &= \zeta^j \hat{B}_{2n, j}( x_1,  x_2, \dots,  x_{2n-j+1})\\
    \hat{B}_{2n, j}(\zeta x_1, \zeta^2 x_2, \dots, \zeta^{2n-j+1} x_{2n-j+1}) &= \zeta^{2n} \hat{B}_{2n, j}( x_1,  x_2, \dots,  x_{2n-j+1})
\end{aligned}
\end{equation}

Из этого следует:
\begin{equation}
\begin{aligned}
    \hat{B}_{2n, j}&\left(-\frac{(2i)^{2}ze^{2iz_k}}{3!}, -\frac{(2i)^{4}ze^{2iz_k}}{4!}, \dots, -\frac{(2i)^{2n-j+2}ze^{2iz_k}}{(2n-j+3)!}\right) = \\
    &= (-2iz e^{2iz_k})^j (2i)^{2n}\hat{B}_{2n, j}\left(\frac{1}{3!}, \frac{1}{4!}, \dots, \frac{1}{(2n-j+3)!}\right)
\end{aligned}
\end{equation}

Используя выше перечисленное\footnote{A также $\Gamma\left(\frac{1}{2}-n\right)= \frac{(2i)^{2n} n!}{(2n)!}\sqrt{\pi}$}, можно написать:

\begin{equation}
\begin{aligned}
    c_{2n} &=
    \sum\limits_{j=0}^{2n} C_{-n-\frac{1}{2}}^j
    \frac{(-2iz e^{2iz_k})^j (2i)^{2n}}{\left(-i - iz e^{2iz_k}\right)^{n+j+\frac{1}{2}}} 
    \frac{j!}{(2n+2j)!}S_{3}(2n+2j, j)\\
    &= \sum\limits_{j=0}^{2n} \frac{n!(2n+2j)!}{(n+j)!(2n)!j! (2i)^{2j}}
    \frac{(-2iz e^{2iz_k})^j (2i)^{2n}}{\left(-i - iz e^{2iz_k}\right)^{n+j+\frac{1}{2}}} 
    \frac{j!}{(2n+2j)!}S_{3}(2n+2j, j)\\
    &= \sum\limits_{j=0}^{2n} \frac{n!}{(n+j)!(2n)!}
    \frac{(-2iz e^{2iz_k})^j (2i)^{2n-2j}}{\left(-i - iz e^{2iz_k}\right)^{n+j+\frac{1}{2}}} S_{3}(2n+2j, j)
\end{aligned}
\end{equation}

\section*{Альтернативное переписывание}
Как можно было заметить, в полученных формулах много некрасивостей. Сейчас, когда мы уже знаем, какие выражения придётся ворочить, предлагается сделать следующие переобозначения:
\begin{equation}
\begin{aligned}
    A &\leadsto \alpha \\
    \Gamma &\leadsto \gamma \\
    Z = R e^{i\Phi}  = -2i A \Gamma&\leadsto -2i\alpha \gamma
\end{aligned}
\end{equation}

Мотивация следующая:
\begin{enumerate}
    \item Большие буквы обозначают неизменность, что важно в контексте множества сумм, парамтров и всего такого
    \item $A$ позволит не путать моё ошибочно выбранное обозначение с общепринятым
    \item $\Gamma$ совпадает с общепринятым обозначением
    \item Большая буква $Z$ обозначает неизменную комплексную величину\footnote{Надеюсь на благоразумие читателей, потому что автор против Z-движения в России}
\end{enumerate}

Так же для упрощения формул с методом перевала немного переписать подынтегральную функцию:
\begin{equation}
\begin{aligned}
    \sum\limits_{n=0}^{\infty} \frac{A^n e^{i\Gamma n^2}}{n!} 
    &= \frac{e^{\frac{i\pi}{4}}}{\sqrt{\pi\Gamma}}
    \int\limits_{-\infty}^{\infty}e^{-i \frac{z^2}{\Gamma} + A e^{2iz}}dz \\
    &= \frac{e^{\frac{i\pi}{4}}}{2\sqrt{\pi\Gamma}}
    \int\limits_{-\infty}^{\infty}e^{-i \frac{z^2}{4\Gamma} + A e^{iz}}dz \\
    &= \frac{e^{\frac{i\pi}{4}}}{2\sqrt{\pi\Gamma}}
    \int\limits_{-\infty}^{\infty}\exp\left(\frac{-i\frac{z^2}{2} + Z i e^{iz}}{2\Gamma}\right)dz 
\end{aligned}
\end{equation}

Сделаем следующее обозначение:
\begin{equation}
\begin{aligned}
    f(z) &=  \frac{z^2}{2i} + i Z e^{iz}
\end{aligned}
\end{equation}

Далее мы будем пользоваться методом перевала. Здесь нам нужно обосновать, что контур действительно можно деформировать ... Этому посвящена моя прога и анализ до этого момента, так что дописать это будет не трудно.

Теперь решения $f'(z_k) = 0$ можно просто обозначить в виде:
\begin{equation}
\begin{aligned}
    &e^{i z_k} = \frac{z_k}{iZ} \Rightarrow
    &z_k = i W_k(Z)
\end{aligned}
\end{equation}

По-моему, такие обозначения просто прекрасны. Вот, например, разложение вокруг перевальной точки:
\begin{equation}
\begin{aligned}
    f(z) = \underbrace{\frac{z_k^2}{2i} + z_k}_{f(z_k)} + \underbrace{\frac{-i-z_k}{2}}_{a_0} (z-z_k)^2 + 
    \sum_{n=1}^{\infty} \underbrace{\frac{-i^n z_k}{(n+2)!}}_{a_1, \: a_2, \dots} (z-z_k)^{n+2}
\end{aligned}
\end{equation}

Полагая $\lambda = \frac{1}{2\Gamma}$ можно написать\footnote{Надеюсь, что читатель не перепутает $\Gamma$ и $\Gamma$-функцию. Для удобства чтения после использования числа $\Gamma$ не будет скобочек.}:
\begin{equation}
\begin{aligned}
    &\int\limits_{\gamma_k} \exp\left(\lambda f(z)\right) dz 
    = \exp\left(\lambda f(z_k)\right)\sum\limits_{n=0}^{\infty} \Gamma\left(n+\frac{1}{2}\right)\frac{c_{2n}}{\lambda^{n+\frac{1}{2}}}\\
    &c_{2n} =
    \sum\limits_{j=0}^{2n} C_{-n-\frac{1}{2}}^j\frac{1}{a_0^{n+j+\frac{1}{2}}}\hat{B}_{2n, j}\left(a_1, a_2, \dots, a_{2n-j+1}\right)
\end{aligned}
\end{equation}
\subsection*{Смелые идеи}
В полученной сумме выделим лидирующий вклад при $n=0$. Он оценивается как $\frac{\sqrt{\pi}}{2\sqrt{\lambda}}\sum\limits_{k=0}^{-\infty}\exp\left(\lambda f(z_k)\right)$. Его можно было бы переписать в виде интеграла, а затем переписать полученное выражение с помощью метода перевала. Это получился бы метод перевала в квадрате! Покажу первые шаги:
\begin{equation}
\begin{aligned}
    \sum\limits_{k=0}^{-\infty}\exp\left(\lambda f(z_k)\right) = 
    \frac{1}{2\pi i}\int\limits_C \frac{e^{\lambda \tilde f(z)}}{-ize^{-iz}-Z} g(z)dz
\end{aligned}
\end{equation}

Контур $C$ необходимо выбрать так, чтоб он обходил все $z_k$. Функции $\tilde f(z)$ и $g(z)$ необходимо выбрать из следующих соображений:
\begin{equation}
\begin{aligned}
    &\tilde f (z_k) = f(z_k) \\
    &\Res\limits_{z=z_k} \frac{e^{\lambda \tilde f(z)}}{-ize^{-iz}-Z} g(z) = e^{\lambda f(z_k)}
\end{aligned}
\end{equation}

Или же, в предположении целости функции $g$:
\begin{equation}
\begin{aligned}
    g(z_k) = \frac{d}{dz}\left(-ize^{-iz}-Z\right)_{z=z_k}  = \underbrace{-(i+z_k)e^{-iz_k}} = -(i+z_k)\frac{iZ}{z_k} = \underbrace{-i(1+Ze^{i z_k})e^{-iz_k}}
\end{aligned}
\end{equation}

Любой из обведённых вариантов удобен для задании функции $g$ в зависимости от поведения модуля подынтегральной функции на комплексной плоскости. Очень важно, чтоб контур интегрирования деформировался удобным образом. Более простой вид для функции $\tilde f$ угадывается следующим образом:
\begin{equation}
\begin{aligned}
    \tilde f(z) = \frac{z^2}{2i} + z = \frac{i}{2} + \frac{(z+i)^2}{2i}
\end{aligned}
\end{equation}

Можно было бы выбрать и другие формы, однако в таком виде очевидным образом находится единственная перевальная точка:
\begin{equation}
\begin{aligned}
    &\tilde z = -i\\
    &\tilde f(\tilde z) = \frac{i}{2}
\end{aligned}
\end{equation}
    

\end{document}