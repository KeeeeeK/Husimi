\documentclass[a4paper, 12pt]{article}

%Русский язык
\renewcommand{\familydefault}{\sfdefault}%шрифт
\usepackage[T2A]{fontenc} %кодировка
\usepackage[utf8]{inputenc} %кодировка исходного кода
\usepackage[english,russian]{babel} %локализация и переносы
%отступы 
\usepackage[left=2cm,right=2cm,top=2cm,bottom=3cm,bindingoffset=0cm]{geometry}
\usepackage{indentfirst}
%Вставка картинок
\usepackage{graphicx}
\graphicspath{}
\DeclareGraphicsExtensions{.pdf,.png,.jpg, .jpeg}

%Таблицы
\usepackage[table,xcdraw]{xcolor}
\usepackage{booktabs}

% Cсылки
\usepackage{hyperref}
%Математика
\usepackage{amsmath, amsfonts, amssymb, amsthm, mathtools }
\DeclareMathOperator*{\Res}{Res}
\DeclareMathOperator*{\sign}{sign}
\DeclareMathOperator*{\Real}{Re}
\DeclareMathOperator*{\Imag}{Im}
%Заголовок
\author{Нугманов Булат}
\title{План доклада}
\begin{document}
\maketitle
\begin{enumerate}
    \item Вступление со значимостью ряда
    \item Переписывание ряда через интеграл
    \item Напоминание о правилах деформирования контуров в комплексном анализе
    \item Напоминание о методе перевала
    \item Поиск перевальных точек и определение в функции Ламберта
    \item Положение ветвей функции Ламберта и особое внимание $W_0$
    \item Кривые постоянной фазы через каждую из точек
    \item График $\Real f(z_k)$ как функции от $k$
    \item Выбор деформации контура в зависимости от знака $\Gamma$
    \item Поправки к методу перевала: числа Белла, ассоциированные числа Стирлинга второго рода третьего типа.
    \item Тупиковый метод с вычетами в точках $z_k$
    \item Правильный метод и выбор $\overline{k}$, асимптотика $W_k(Z)$
    \item Максимум по $\Phi$
\end{enumerate}
\section*{Оставшиеся вопросы}
\begin{enumerate}
    \item Разложение по малости $\Gamma$, полученное как остатки от метода перевала является, скорее всего, расходящимся. Это вещь обыкновенная проблема и борьба с ней лишь одна: надо хотя бы оценить $n_{max}$ до которого позволительно вести суммирование
    \item Всё из того же разложения по методу перевала необходимо получить поправку к точке максимума. В действительности, как следует из численных расчётов, должна быть поправка на $\Gamma$.
    \item Для больших $Z$ есть асимптотическая формула для оптимального $\bar k$. Необходимо получить численные расчёты для меньших $Z$, где асимптотика работает плохо.
    \item Я так и не получил чудо-формулы из диплома для аппроксимации $F$. Кроме уже упомянутой точки максимума необходимо отловить малые поправки, которые будут приводить к второй производной в окрестности точки максимума. Это... посильно, но требует некоторых, пока неясных действий.
    \item А ещё есть очень интересно описываемая асимптотика не только при $|A|\Gamma\sim 1$, но и при $|A|\sqrt{\Gamma}\sim 1$... Я её случайно находил, но в диплом не вставлял. Там тоже работает какая-то аналитика и рискну предположить, что вся она крутиться вокруг разложения $W_0(Z)$ в ряд по малости $Z$. 
\end{enumerate}

Труднее всего, как мне кажется, найти поправку к точке максимума. Как показывают численные расчёты, даже учёт слагаемых порядка $\frac{L2}{L1^4}$ не приводит к настолько хорошим аппроксимациям, как формула из диплома.

Если готовить на продажу без всякой асимптотики из диплома, то пункты 1 и 3 остро необходимы для максимально точных численных расчётов.
\end{document}