\documentclass[a4paper, 12pt]{article}

%Русский язык
\renewcommand{\familydefault}{\sfdefault}%шрифт
\usepackage[T2A]{fontenc} %кодировка
\usepackage[utf8]{inputenc} %кодировка исходного кода
\usepackage[english,russian]{babel} %локализация и переносы
%отступы 
\usepackage[left=2cm,right=2cm,top=2cm,bottom=3cm,bindingoffset=0cm]{geometry}
\usepackage{indentfirst}
%Вставка картинок
\usepackage{graphicx}
\graphicspath{}
\DeclareGraphicsExtensions{.pdf,.png,.jpg, .jpeg}

%Таблицы
\usepackage[table,xcdraw]{xcolor}
\usepackage{booktabs}

% Cсылки
\usepackage{hyperref}
\bibliographystyle{unsrt}
%Математика
\usepackage{amsmath, amsfonts, amssymb, amsthm, mathtools }
\DeclareMathOperator*{\Res}{Res}
\DeclareMathOperator*{\sign}{sign}
\DeclareMathOperator*{\Real}{Re}
\DeclareMathOperator*{\Imag}{Im}

%%Окружение для многострочных уравнений
\newenvironment{eqw}{\begin{equation} \begin{aligned}}   
    {\end{aligned}    \end{equation}}
\newenvironment{eqw*}{\begin{equation*} \begin{aligned}}   
    {\end{aligned}    \end{equation*}}
%Заголовок
\author{Нугманов Булат}
\title{Статья}
\begin{document}
\maketitle
\section*{План}
\begin{itemize}
    \item Вступление 
    \begin{enumerate}
        \item Вакуум генерирует когерентные состояния
        \item Важность генерации некогерентные состояний: ссылки на quantum computation, квантовую связь ... \textcolor{red}{на что ещё?}
        \item Основные механизмы генерации некогерентные состояний: нелинейная среда ... \textcolor{red}{что-нибудь ещё?}
        \item Использование нелинейных кристаллов: $\chi^{(2)}$, $\chi^{(3)}$. \textcolor{red}{какие-нибудь ссылки на использование высших порядков нелинейности}
        \item Меры неклассичности: соотношение между дисперсией и средним числом фотонов
        \item $\chi^{(2)}$ не придаёт негативности, но позволяет снизить дисперсию. $\chi^{(3)}$ даёт негативность. Подмешивание с гауссовым светом даёт меньшую дисперсию \textcolor{red}{А можно ли мешать не с гауссовым светом, а с чем-то иным?}. Ссылка на китагаву и перечисленных в BalKhal ещё авторов
        \item Использование $\chi^{(3)}$ --- в резонаторе и напрямую через кристалл. 
        \item Хорошо бы ещё оформить в таблицу экспериментальные достижения из BalKhal и более новые исследования. Одним из столбцов можно сделать особенности эксперимента или практического применения.
    \end{enumerate}
    \item Основная часть
    \begin{enumerate}
        \item Гамильтониан света в кристалле, пренебрежение остальными модами. Разложение получившегося состояния по фоковским
        \item Изучение функции Хусими гораздо проще, чем функции Вигнера, потому что её можно свести к зависимости от двух парамтров
        \item Использование введённой функции $F$ + ссылка на Supplemental material = графики функции Хусими при различных $\alpha, \; \Gamma$.
        \item Обсуждение основных параметров, от которых зависит картинка. Закручивание функции Хусими на полный поворот --- \textit{вполне достижимая картинка в эксперименте?}
    \end{enumerate}
    \item Заключение\\
    Бла-бла. Мы всё сделали, всё получилось.
\end{itemize}
\end{document}