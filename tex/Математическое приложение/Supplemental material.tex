\documentclass[a4paper, 12pt]{article}

%Русский язык
\renewcommand{\familydefault}{\sfdefault}%шрифт
\usepackage[T2A]{fontenc} %кодировка
\usepackage[utf8]{inputenc} % кодировка исходного кода
\usepackage[english,russian]{babel} %локализация и переносы
%отступы 
\usepackage[left=2cm,right=2cm,top=2cm,bottom=3cm,bindingoffset=0cm]{geometry}
\usepackage{indentfirst}
%Вставка картинок
\usepackage{graphicx}
\graphicspath{}
\DeclareGraphicsExtensions{.pdf,.png,.jpg, .jpeg}

%Таблицы
\usepackage[table,xcdraw]{xcolor}
\usepackage{booktabs}

% Ссылки
\usepackage{hyperref}
%Математика
\usepackage{amsmath, amsfonts, amssymb, amsthm, mathtools }
\DeclareMathOperator*{\Res}{Res}
\DeclareMathOperator*{\sign}{sign}
\DeclareMathOperator*{\Real}{Re}
\DeclareMathOperator*{\Imag}{Im}

%%Окружение для многострочных уравнений
\newenvironment{eqw}{\begin{equation} \begin{aligned}}   
    {\end{aligned}    \end{equation}}
\newenvironment{eqw*}{\begin{equation*} \begin{aligned}}   
    {\end{aligned}    \end{equation*}}
%Заголовок
\author{Нугманов Булат}
\title{Математическое приложение}
\begin{document}
\maketitle
Данный документ является наброском математического приложения, результаты которого точно выполняются численно. А ещё надо будет это перевести.

\begin{enumerate}
    \item Переписывание ряда через интеграл
    \item Нахождение перевальных точек
    \item Асимптотика кривых постоянной фазы через перевальные точки --- ???
    \item Деформация контура в зависимости от знака $\Gamma$
    \item Подсчёт вкладов по методу перевала
    \item Вставка с полиномами Белла и числами Стирлинга для явного подсчёта коэффициентов в методе перевала
    \item Слова о том, что при $|\Gamma| \ll 1$ достаточно учитывать лишь одно слагаемое
    \item Точное значение в максимуме
    \item Выбор $\overline{k}$
    \item Граница выбора $\overline{k}$
    \item Итог: асимптотика ряда
\end{enumerate}
\section*{Вступление}
Целью данного документа является подсчёт асимптотики ряда следующего вида:
\begin{eqw}
    \sum\limits_{n=0}^\infty
\end{eqw}

\section*{Переход от суммы к интегралу}



\begin{eqw}
    \sum\limits_{n=0}^{\infty} \frac{\alpha^n e^{i\gamma n^2}}{n!} 
    &= \frac{e^{\frac{i\pi}{4}}}{\sqrt{\pi\gamma}}
    \int\limits_{-\infty}^{\infty}e^{-i \frac{x^2}{\gamma} + \alpha e^{2ix}}dx
    &= \frac{e^{\frac{i\pi}{4}}}{\sqrt{\pi\gamma}}
    \int\limits_{-\infty}^{\infty}e^{\frac{1}{\gamma}\left(-i x^2 + \alpha \gamma e^{2ix}\right)}dx
\end{eqw}
\end{document}