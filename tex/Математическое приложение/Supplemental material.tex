\documentclass[a4paper, 12pt]{article}

%Русский язык
\renewcommand{\familydefault}{\sfdefault}%шрифт
\usepackage[T2A]{fontenc} %кодировка
\usepackage[utf8]{inputenc} % кодировка исходного кода
\usepackage[english,russian]{babel} %локализация и переносы
%отступы 
\usepackage[left=2cm,right=2cm,top=2cm,bottom=3cm,bindingoffset=0cm]{geometry}
\usepackage{indentfirst}
%Вставка картинок
\usepackage{graphicx}
\graphicspath{}
\DeclareGraphicsExtensions{.pdf,.png,.jpg, .jpeg}

%Таблицы
\usepackage[table,xcdraw]{xcolor}
\usepackage{booktabs}

% Ссылки
\usepackage{hyperref}
%Математика
\usepackage{amsmath, amsfonts, amssymb, amsthm, mathtools }
\DeclareMathOperator*{\Res}{Res}
\DeclareMathOperator*{\sign}{sign}
\DeclareMathOperator*{\Real}{Re}
\DeclareMathOperator*{\Imag}{Im}

%%Окружение для многострочных уравнений
\newenvironment{eqw}{\begin{equation} \begin{aligned}}   
    {\end{aligned}    \end{equation}}
\newenvironment{eqw*}{\begin{equation*} \begin{aligned}}   
    {\end{aligned}    \end{equation*}}
%Заголовок
\author{Нугманов Булат}
\title{Математическое приложение}
\begin{document}
\maketitle
Данный документ является наброском математического приложения, результаты которого точно выполняются численно. А ещё надо будет это перевести.

\begin{enumerate}
    \item Переписывание ряда через интеграл
    \item Нахождение перевальных точек
    \item Асимптотика кривых постоянной фазы через перевальные точки --- ???
    \item Деформация контура в зависимости от знака $\Gamma$
    \item Подсчёт вкладов по методу перевала
    \item Вставка с полиномами Белла и числами Стирлинга для явного подсчёта коэффициентов в методе перевала
    \item Слова о том, что при $|\Gamma| \ll 1$ достаточно учитывать лишь одно слагаемое
    \item Точное значение в максимуме
    \item Выбор $\overline{k}$
    \item Граница выбора $\overline{k}$
    \item Итог: асимптотика ряда
\end{enumerate}
\section*{Вступление}
Целью данного документа является подсчёт асимптотики ряда следующего вида:
\begin{eqw}
    \sum\limits_{n=0}^{\infty} \frac{A^n e^{i\Gamma n^2}}{n!} 
\end{eqw}

\section*{Переход от суммы к интегралу}
Первым делом избавимся от $\exp\left(i\Gamma n^2\right)$
\begin{eqw}
    \exp\left(i\Gamma n^2\right) = \frac{e^{i\frac{\pi}{4}\sign \Gamma}}{2\sqrt{\pi|\Gamma|}}
    \int\limits_{-\infty}^{\infty} \exp\left(-\frac{z^2}{4\Gamma} + i n z\right) dz
\end{eqw}

\begin{eqw}
    \sum\limits_{n=0}^{\infty} \frac{A^n e^{i\Gamma n^2}}{n!} = \frac{e^{i\frac{\pi}{4}\sign \Gamma}}{2\sqrt{\pi|\Gamma|}}
    \int\limits_{-\infty}^{\infty} \exp\left(-\frac{z^2}{4\Gamma} + i A z\right) dz
\end{eqw}

\begin{eqw}
    Z = -2i A \Gamma = R e^{i\Phi}
\end{eqw}

\begin{eqw}
    \sum\limits_{n=0}^{\infty} \frac{A^n e^{i\Gamma n^2}}{n!} = \frac{e^{i\frac{\pi}{4}\sign \Gamma}}{2\sqrt{\pi|\Gamma|}}
    \int\limits_{-\infty}^{\infty}\exp\left(\frac{-i\frac{z^2}{2} + i Z e^{iz}}{2\Gamma}\right)dz 
\end{eqw}

\begin{eqw}
    f(z) =  \frac{z^2}{2i} + i Z e^{iz}
\end{eqw}

\begin{eqw}
    e^{i z_k} = \frac{z_k}{iZ} \Rightarrow   z_k = i W_k(Z)
\end{eqw}

\begin{eqw}
    f(z) = \underbrace{\frac{z_k^2}{2i} + z_k}_{f(z_k)} + \underbrace{\frac{-i-z_k}{2}}_{a_0} (z-z_k)^2 + 
    \sum_{n=1}^{\infty} \underbrace{\frac{-i^n z_k}{(n+2)!}}_{a_1, \: a_2, \dots} (z-z_k)^{n+2}
\end{eqw}

Считая $1$:
\begin{eqw}
    \int\limits_{\gamma_k} \exp\left(\frac{f(z)}{2\Gamma}\right) dz 
    = \exp\left(\frac{ f(z_k)}{2\Gamma} \right)\sum\limits_{n=0}^{\infty} \Gamma\left(n+\frac{1}{2}\right)\left(2\Gamma\right)^{n+\frac{1}{2}} \sum\limits_{j=0}^{2n} \frac{C_{-n-\frac{1}{2}}^j}{a_0^{n+j+\frac{1}{2}}}\hat{B}_{2n, j}\left(a_1, a_2, \dots, a_{2n-j+1}\right)
\end{eqw}
\end{document}