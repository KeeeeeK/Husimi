\documentclass[a4paper, 12pt]{article}

%Русский язык
\renewcommand{\familydefault}{\sfdefault}%шрифт
\usepackage[T2A]{fontenc} %кодировка
\usepackage[utf8]{inputenc} % кодировка исходного кода
\usepackage[english,russian]{babel} %локализация и переносы
%отступы 
\usepackage[left=2cm,right=2cm,top=2cm,bottom=3cm,bindingoffset=0cm]{geometry}
\usepackage{indentfirst}
%Вставка картинок
\usepackage{graphicx}
\graphicspath{}
\DeclareGraphicsExtensions{.pdf,.png,.jpg, .jpeg}
%Таблицы
\usepackage[table,xcdraw]{xcolor}
\usepackage{booktabs}
% Ссылки
\usepackage{hyperref}
% Цветной текст
\usepackage{xcolor}
%Математика
\usepackage{amsmath, amsfonts, amssymb, amsthm, mathtools }
\DeclareMathOperator*{\Res}{Res}
\DeclareMathOperator*{\sign}{sign}
\DeclareMathOperator*{\Real}{Re}
\DeclareMathOperator*{\Imag}{Im}

%%Окружение для многострочных уравнений
\newenvironment{eqw}{\begin{equation} \begin{aligned}}   
    {\end{aligned}    \end{equation}}
\newenvironment{eqw*}{\begin{equation*} \begin{aligned}}   
    {\end{aligned}    \end{equation*}}
%Заголовок
\author{Нугманов Булат}
\title{Математическое приложение}
\begin{document}
\maketitle
Данный документ является наброском математического приложения, результаты которого точно выполняются численно. А ещё надо будет это перевести.

\begin{enumerate}
    \item Переписывание ряда через интеграл
    \item Нахождение перевальных точек
    \item Асимптотика кривых постоянной фазы через перевальные точки --- \textcolor{red}{не знаю, надо ли об этом писать?}
    \item Деформация контура в зависимости от знака $\Gamma$ --- \textcolor{red}{пока нет}
    \item Подсчёт вкладов по методу перевала
    \item Вставка с полиномами Белла и числами Стирлинга для явного подсчёта коэффициентов в методе перевала
    \item Слова о том, что при $|\Gamma| \ll 1$ достаточно учитывать лишь одно слагаемое
    \item Точное значение в максимуме
    \item Выбор $\overline{k}$
    \item Граница выбора $\overline{k}$
    \item Итог: асимптотика ряда
\end{enumerate}
\section*{Вступление}
In this section we will find the asymptotics of the series of the following form:
\begin{eqw}\label{F def}
    F(A, e^{i\Gamma}) = \sum\limits_{n=0}^{\infty} \frac{A^n e^{i\Gamma n^2}}{n!},
\end{eqw}
when $\Gamma\in \mathbb{R}$, $A\in \mathbb{C}$, $|\Gamma| \ll 1$ and $|A\Gamma|\gtrsim 1$. 

Наметим примерный план дальнейшего рассказа. Первым делом мы перейдём от суммы к расчёту интеграла по действительной оси. Далее мы воспользуемся отсутствием полюсов подынтегральной функции и деформируем контур так, чтоб он проходил через перевальные точки по кривым постоянной фазы. Перевальных точек будет бесконечное количество, однако наиболее существенный вклад будет вносить интеграл только одна из кривых постоянной фазы. Для пущей точности мы найдём все остаточные члены метода перевала. 
\section*{Переход от суммы к интегралу}
Вычисление суммы "табличными" методами существенно осложняется множителем $\exp\left(i\Gamma n^2\right)$. Воспользуемся следующей формулой (здесь и далее мы будем подразумевать все интегралы в смысле главного значения):
\begin{eqw}
    \exp\left(i\Gamma n^2\right) = \frac{e^{i\frac{\pi}{4}\sign \Gamma}}{2\sqrt{\pi|\Gamma|}}
    \int\limits_{-\infty}^{\infty} \exp\left(-\frac{z^2}{4\Gamma} + i n z\right) dz
\end{eqw}
Применим данную формулу для ряда \ref{F def}:
\begin{eqw}
    \sum\limits_{n=0}^{\infty} \frac{A^n e^{i\Gamma n^2}}{n!} = \frac{e^{i\frac{\pi}{4}\sign \Gamma}}{2\sqrt{\pi|\Gamma|}}
    \int\limits_{-\infty}^{\infty} \exp\left(-\frac{z^2}{4\Gamma} + i A z\right) dz
\end{eqw}
Приведём данную формулу к более удобному для применения метода перевала виду. Для этого введём обозначения
\begin{eqw}
    f(z) =  \frac{z^2}{2i} + i Z e^{iz}\\
    Z = -2i A \Gamma = R e^{i\Phi},
\end{eqw}
где $\Phi\in\left(-\pi, \pi\right]$ и $R\in\mathbb{R}$. И наконец, 
\begin{eqw}
    \sum\limits_{n=0}^{\infty} \frac{A^n e^{i\Gamma n^2}}{n!} = \frac{e^{i\frac{\pi}{4}\sign \Gamma}}{2\sqrt{\pi|\Gamma|}}
    \int\limits_{-\infty}^{\infty}\exp\left(\frac{f(z)}{2\Gamma}\right)dz 
\end{eqw}

\section*{Применение метода перевала}
\subsection*{Перевальные точки}
Начнём с поиска перевальных точек, которые являются решением $f'(z)=0$. Бесконечное множество таких решений можно занумеровать номером $k$:
\begin{eqw}
    e^{i z_k} = \frac{z_k}{iZ} \Rightarrow   z_k = i W_k(Z),
\end{eqw}
где $W_k$ --- $k$-ая ветвь $W$-функции Ламберта. 
\subsection*{Деформация контура через кривые постоянной фазы}
\textcolor{red}{Здесь должно быть объяснение того, почему мы выбриаем только некоторые из всего множества точек. График с положением перевальных точек, кривыми постоянных фаз через каждую из точек, доказательство того, что номера выбора перевальных точек должны быть инвариантны относительно $Z$ и пара слов про расходимость теории при особом случае $Z=-e^{-1}$. А ещё пара слов о том, почему контур действительно деформируется и удалённых областях интеграл по дополнительным (непервальным) контурам стремится к нулю.}

Правильно выбрав контура интегрирования $\gamma_k$, мы можем сформулировать следующее:
\begin{eqw}
    \int\limits_{-\infty}^{\infty}\exp\left(\frac{f(z)}{2\Gamma}\right) dz = \sum\limits_{k=0}^{-\sign\Gamma \cdot \infty}
    \int\limits_{\gamma_k} \exp\left(\frac{f(z)}{2\Gamma}\right) dz
\end{eqw}
\subsection*{Интеграл по $\gamma_k$}
Для расчёта интеграла вдоль некоторой $\gamma_k$, мы воспользуемся формулой CFWW \textcolor{red}{вставка книги по методу перевала}. Для её применения нам необходимо разложение в ряд Тейлора функции $f$ в окрестности перевальной точки:
\begin{eqw}
    f(z) = \underbrace{\frac{z_k^2}{2i} + z_k}_{f(z_k)} + \underbrace{\frac{-i-z_k}{2}}_{a_0} (z-z_k)^2 + 
    \sum_{n=1}^{\infty} \underbrace{\frac{-i^n z_k}{(n+2)!}}_{a_1, \: a_2, \dots} (z-z_k)^{n+2}
\end{eqw}
\section*{Подсчёт вкладов по методу перевала}
\begin{eqw}
    \int\limits_{\gamma_k} \exp\left(\frac{f(z)}{2\Gamma}\right) dz 
    &= \exp\left(\frac{ f(z_k)}{2\Gamma} \right)\sqrt{\frac{2\Gamma}{a_0}}\sum\limits_{n=0}^{\infty} 
    \Gamma\left(n+\frac{1}{2}\right)\left(2\Gamma\right)^n c_{2n}\\
    c_{2n} &= \sum\limits_{j=0}^{2n} \frac{C_{-n-\frac{1}{2}}^j}{a_0^{n+j}}\hat{B}_{2n, j}\left(a_1, a_2, \dots, a_{2n-j+1}\right)
\end{eqw}
\section*{Полезная формула}
Итогом должно стать:
\begin{eqw}
    \hat{B}_{n, k}\left(\frac{1}{r!}, \frac{1}{(r+1)!}, \dots, \frac{1}{(n-k+r)!}\right) = 
    \frac{k!}{(n+(r-1)k)!}S_{r}(n+(r-1)k, k)
\end{eqw}
где далее можно использовать
\begin{eqw}
    S_r(n+1, k) = k S_r(n, k) + C_n^{r} S_r(n-r+1, k-1)
\end{eqw}
\subsection*{Полиномы Белла}
\begin{eqw}
    \exp\left(u \sum\limits_{j=1}^{\infty}x_j t^j\right) = \sum\limits_{n\geq k\geq 0} 
    \hat{B}_{n,k}\left(x_1, x_2, \dots, x_{n-k+1}\right) t^n \frac{u^k}{k!}
\end{eqw}

\begin{eqw}
    \hat{B}_{2n, j}(\zeta x_1, \zeta x_2, \dots, \zeta x_{2n-j+1}) &= \zeta^j \hat{B}_{2n, j}( x_1,  x_2, \dots,  x_{2n-j+1})\\
    \hat{B}_{2n, j}(\zeta x_1, \zeta^2 x_2, \dots, \zeta^{2n-j+1} x_{2n-j+1}) &= \zeta^{2n} \hat{B}_{2n, j}( x_1,  x_2, \dots,  x_{2n-j+1})
\end{eqw}
В нашем случае:
\begin{eqw}
    \hat{B}_{2n, j}\left(a_1, a_2, \dots, a_{2n-j+1}\right) &= 
    \hat{B}_{2n, j}\left(\frac{-i z_k}{3!}, \frac{-i^2 z_k}{4!}, \dots, \frac{-i^{2n-j+1} z_k}{(2n-j+3)!}\right) \\
    &= (-1)^n(-z_k)^{j}\hat{B}_{2n, j}\left(\frac{1}{3!}, \frac{1}{4!}, \dots, \frac{1}{(2n-j+3)!}\right)
\end{eqw}

\subsection*{Числа Стирлинга}
\begin{eqw}
    \exp\left(u\left(\frac{t^r}{r!}+\frac{t^{r+1}}{\left(r+1\right)!}+\dots\right)\right) = \sum\limits_{n=(r+1)k, k = 0}^{\infty} S_{r}(n, k) u^k \frac{t^n}{n!}
\end{eqw}

Сравнивая данную строчку с определением полиномов Белла,
\begin{eqw}
    \sum\limits_{n\geq k\geq 0} 
    \hat{B}_{n, k}\left(\frac{1}{r!}, \frac{1}{(r+1)!}, \dots, \frac{1}{(n-k+r)!}\right) t^n \frac{u^k}{k!} &= 
    \exp\left(\frac{u}{t^{r-1}} \left(\frac{t^r}{r!} + \frac{t^{r+1}}{(r+1)!}+\dots \right)\right) \\ 
    &= \sum\limits_{n, k}^{\infty} S_{r}(n, k) \frac{u^k}{t^{(r-1)k}}\frac{t^n}{n!} \\
    &= \sum\limits_{n, k}^{\infty} S_{r}(n+(r-1)k, k) u^k\frac{t^n}{(n+(r-1)k)!}
\end{eqw}


\section*{Наилучшее $\bar k$}
Используя всё выше перечисленное и обозначение для $z_k$:
\begin{eqw}
    \sum\limits_{n=0}^{\infty}\left(\frac{iZ}{2\Gamma}\right)^n  \frac{e^{i\Gamma n^2}}{n!} &= e^{\frac{i\pi}{4}}
    \sum\limits_{k=0}^{-\sign\Gamma\cdot\infty} \frac{\exp\left(\frac{-i-i(z_k+i)^2}{4\Gamma}\right)}{(-i-z_k)^{\frac{1}{2}}}
    \sum\limits_{n=0}^{\infty}\left(2\Gamma\right)^n
    \sum\limits_{j=0}^{2n}\left(-\frac{1}{2}\frac{z_k}{i+z_k}\right)^{n+j}\frac{S_3(2n+2j, j)}{(n+j)!},
\end{eqw}
Далее мы переходим к менее точным оценкам.
\begin{eqw}\label{approx_exact_row}
    \sum\limits_{n=0}^{\infty}\left(\frac{iZ}{2\Gamma}\right)^n  \frac{e^{i\Gamma n^2}}{n!} \approx 
    e^{\frac{i\pi}{4}} \frac{\exp\left(\frac{-i-i(z_{\bar k}+i)^2}{4\Gamma}\right)}{(-i-z_{\bar k})^{\frac{1}{2}}}
\end{eqw}
Выпишем действительную часть показателя экспоненты:
\begin{eqw}
    \Real\left(f(z_k)\right) = \frac{1}{2}\Real\left(-i(z_k+i)^2\right) = -\Imag(W_k(Z))\left(1+\Real(W_k(Z))\right)
\end{eqw}
\subsection*{Точное поведение $\Real f(z_k)$}
Найдём поведение $\Real f(z_k)\sign \Gamma$ как функцию от $\Phi$ в окрестности точки максимума. Для сокращения дальнейших формул используем в данном параграфе сокращение $W = W_k(Z=R e^{i\Phi})$. Для этого нам понадобиться:

\begin{eqw}
    \frac{\partial W}{\partial \Phi} = \frac{iW}{1+W}
\end{eqw}

\begin{eqw}
    \frac{\partial}{\partial\Phi}\Real\left(f(z_k)\right) = -\Real(W)
\end{eqw}

В точке максимума $\frac{\partial}{\partial\Phi}\Real\left(f(z_k)\right) = 0$. Отсюда из определения $W$-функции Ламберта
\begin{eqw}
    i\Imag W e^{i\Imag W} = R e^{i\Phi_{\max}} \Rightarrow \left|\Imag W\right| = R
\end{eqw}

\begin{eqw}
    \sign\left(\frac{f(z_k)}{2\Gamma}\right)_{\Phi = \Phi_{max}} = 1 \Rightarrow \Imag W = R\sign k 
    &\Rightarrow \left(\frac{f(z_k)}{2\Gamma}\right)_{\Phi = \Phi_{max}} = \frac{R}{2|\Gamma|}
\end{eqw}

\begin{eqw}
    i R \sign k  e^{i R \sign k } = R e^{i\Phi_{\max}} &\Rightarrow \Phi_{\max} = \left(R+\frac{\pi}{2}\right)\sign k \;\;\; \mod 2\pi
\end{eqw}
Взятием производных более высоких порядков можно получить, что:

\begin{eqw}
    \Real f(z_k) = R\sign \Gamma - \sum\limits_{m=1}^{\infty} \Real\left(\frac{i^m q_{m}\left(-iR\sign \Gamma\right)}{\left(1-iR\sign \Gamma\right)^{2m-1}}\right)\frac{\left(\Phi - \Phi_{\max}\right)^{m+1}}{(m+1)!}
\end{eqw}
\begin{eqw}
    \Real f(z_k) = R\sign \Gamma + \sum\limits_{m=1}^{\infty} \frac{\left(\Phi - \Phi_{\max}\right)^{m+1}}{(m+1)!}
    \sum\limits_{k=0}^{m-1}\left\langle\left\langle{m-1}\atop {k}\right\rangle\right\rangle 
    \Real\left(\frac{i^{m}\left(i R\sign \Gamma\right)^{k+1} }{\left(1-iR\sign\Gamma\right)^{2m-1}},\right)
\end{eqw}
где принято обозначение $\left\langle\left\langle{m-1}\atop {k}\right\rangle\right\rangle$ --- Эйлеровы числа второго рода
\subsection*{Асимптотика $W$ при $R \to \infty$}
Несмотря на заявленное в заголовке, асимптотика хорошо работает и при небольших $R$.
\begin{eqw}
    W_k(z) &= \log z + 2\pi i k - \log\left(\log z + 2\pi i k\right) + \sum\limits_{k=0}^{\infty}\sum\limits_{m=1}^{\infty} c_{km}\log^m\left(\log z + 2\pi i k\right)\left(\log z + 2\pi i k \right)^{-k-m}\\
    c_{km} &= \frac{(-1)^k}{m!} c(k+m, k+1),
\end{eqw}
где $c(k+m, k+1)$ ---  беззнаковые числа Стирлинга первого рода.

% В первом приближении:
% \begin{eqw}
%     W_{k}\left(Z=R e^{i\Phi}\right) &\approx \log R + i\Phi + 2\pi i k \\
%     \Real\left(f(z_k)\right) &\approx -(2\pi k+\Phi)(1+\log R)
% \end{eqw}
% Из данного разложения находим: $\bar k\to\infty$, при $R\to\infty$

% Попробуем учесть следующий порядок. В разложении будем считать, что $|k|\gg \log R$:

Пренебрегая членами $O\left(\frac{\log\left(\log R + 2\pi i k\right)}{\log R  + 2\pi i k}\right)$ в разложении функции Ламберта:
\begin{eqw}
    W_k\left(Z\right) &\approx \log R + i\Phi + 2\pi i k - \log\left(\log R + i\Phi+ 2\pi i k\right) \\
    &\approx \log R - \frac{1}{2}\log\left(\log^2 R + (2\pi k)^2\right) + 
    i\left(\Phi + 2\pi k - \arctan\left(\frac{2\pi k}{\log R}\right)\right) \\
    \Real\left(f(z_k)\right) &\approx
    -\left(\Phi + 2\pi k - \arctan\left(\frac{2\pi k}{\log R}\right)\right)\left(1-
    \frac{1}{2}\log \left(\left(\frac{2\pi k}{R}\right)^2+\frac{\log^2 R}{R^2}\right)
    \right)
\end{eqw}

Из этого выражения мы видим, что $\bar k = O(R)$. Более детальный расчёт максимума последнего выражения по $k$ приводит к:

\begin{eqw}\label{optimal_k}
    |\bar k|\approx \left[ \frac{1}{2\pi}\left(R + \left|\Phi + \frac{\pi}{2}\sign \Gamma\right|\right)\right]
\end{eqw}

Эта формула однозначно делит комплексную плоскость для $Z$ на области выбора оптимального $\bar k$. Разложение $W_{k}(Z)$ с точностью до $O\left(\frac{\log^3 R}{R^3}\right)$ приводит к следующей формуле:
\begin{eqw}
    \Real(f(z_k)) =  R\sign \Gamma + \frac{\left(\Phi + 2\pi \bar k +  \left(R+\frac{\pi}{2} \right)\sign\Gamma\right)^2}{2R} +
     O\left(\frac{\log^3 R}{R^2}\right)
\end{eqw}

Решение уравнения $\Real(f(z_{k-\sign \Gamma})) = \Real(f(z_{k}))$ приводит к тому, что оптимум $\bar k$, найденный в уравнении \ref{optimal_k}, имеет точность определения границы областей оптимального $\bar k$ на комплексной плоскости для $Z$ порядка $O\left(\frac{\log^3 R}{R}\right)$.

\section*{Финал}
Таким образом, можно получать либо достаточно точную асимптотику из выражения \ref{approx_exact_row}, либо, при желании сэкономить на расчёте функции Ламберта, можно использовать более наглядную формулу для модуля в окрестности точки максимума:

\begin{eqw}
    \left|\sum\limits_{n=0}^{\infty}\left(\frac{iZ}{2\Gamma}\right)^n  \frac{e^{i\Gamma n^2}}{n!} \right|\approx 
    \frac{1}{\sqrt{R}}\left(1-\frac{1}{4 R^2} + \frac{5}{32 R^4}+\dots\right)
    \exp\left(\frac{R}{2|\Gamma|} - \frac{\left(\delta -\Gamma\right)^2}{4R|\Gamma|}+\dots\right),
\end{eqw}
где ввели обозначение $\delta = \Phi + 2\pi \bar k +  \left(R+\frac{\pi}{2} \right)\sign\Gamma$.

\end{document}