\documentclass[a4paper, 12pt]{article}

%Русский язык
\renewcommand{\familydefault}{\sfdefault}%шрифт
\usepackage[T2A]{fontenc} %кодировка
\usepackage[utf8]{inputenc} %кодировка исходного кода
\usepackage[english,russian]{babel} %локализация и переносы
%отступы 
\usepackage[left=2cm,right=2cm,top=2cm,bottom=3cm,bindingoffset=0cm]{geometry}
\usepackage{indentfirst}
%Вставка картинок
\usepackage{graphicx}
\graphicspath{}
\DeclareGraphicsExtensions{.pdf,.png,.jpg, .jpeg}

%Таблицы
\usepackage[table,xcdraw]{xcolor}
\usepackage{booktabs}


%Математика
\usepackage{amsmath, amsfonts, amssymb, amsthm, mathtools }
\DeclareMathOperator{\Real}{Re}
%Заголовок
\author{Нугманов Булат}

\begin{document}
\section*{Приближение для ряда}
Обозначение:
\begin{equation}
\begin{aligned}
    \alpha = r e^{i\phi}
\end{aligned}
\end{equation}

Стирлинг:
\begin{equation}
\begin{aligned}
    n! \approx \sqrt{2\pi n} \left(\frac{n}{e}\right)^n e^{\frac{1}{12n}+\dots}
\end{aligned}
\end{equation}

Логарифм одного слагаемого в $F-normalized$:
\begin{equation}
\begin{aligned}
    &\ln\left(\frac{\alpha^n e^{i\phi n + i\gamma n(n+1)}}{n!}e^{-r}\right)  \approx 
    \ln\left(\frac{r^n e^{i\phi n + i\gamma n(n+1)}}{\sqrt{2\pi n} \left(\frac{n}{e}\right)^n e^{\frac{1}{12n}}}e^{-r}\right) =\\
    &=n \left(\ln r + i \phi + i \gamma (n+1)\right) - \frac{1}{2}\ln\left(2\pi\right) 
    - \frac{1}{2}\ln n - n \left(\ln n - 1\right) - \frac{1}{12n} - r\\
    &=n-r + n\cdot\ln\left(\frac{r}{n}\right) + i n \left(\phi + \gamma(n+1)\right)
     -  \frac{1}{12n} - \frac{1}{2}\ln n - \frac{1}{2}\ln\left(2\pi\right) 
\end{aligned}
\end{equation}

\section*{Из в $F$ в Хусими}
Это формула 3.3 из моего диплома:
\begin{equation}\label{QsimF}
	Q(\beta) =
	\frac{e^{-|\alpha|^2 -|\beta|^2}}{\pi} \left|F(\alpha \beta^* e^{2 i \Gamma}, e^{-i \Gamma} )\right|^2
\end{equation}

Если использовать $Fn$ --- $F\; normalized$, которое $Fn(r, \phi, \gamma) = F(r e^{i\phi}, e^{i\gamma}) \cdot e^{-r}$, то формула предстанет в следующем виде:

\begin{equation}
\begin{aligned}
    Q(\beta) = \frac{e^{-\left(|\alpha|-|\beta|\right)^2}}{\pi}\left|Fn(|\alpha\beta|, 2 \Gamma + \arg\left(\alpha\beta^*\right), -\Gamma)\right|^2
\end{aligned}
\end{equation}

\section*{Вычисление функции Вигнера}
Возьмём некоторый алгоритм расчёта функции $F$, указанной в дипломе. Это может быть как прямое суммирование, так и нахождение через $\max \Real f(z_k)$ или же ассимптотическое разложение при $|Z|\gg 1$. Для вычисления функции Вигнера лучше использовать формулу из диплома под номером (3.25):

\begin{equation}\label{FtoWrow}
	\begin{aligned}
		W(\beta) = \frac{2}{\pi}e^{-|\alpha|^2-2|\beta|^2}\sum\limits_{m=0}^{\infty} \frac{\left(-|\alpha|^2\right)^m}{m!}\left|F(2\alpha \beta^*\psi^{2m-2}, \psi)\right|^2
	\end{aligned}
\end{equation}

Перепишем эту формулу через $Fn(|A|, \arg A, e^{i\Gamma}) = e^{-|A|}F(A, e^{i\Gamma})$:
\begin{equation}
	\begin{aligned}
		W(\beta) = \frac{2}{\pi}e^{-|\beta|^2 -\left(|\alpha|-|\beta|\right)^2}\sum\limits_{m=0}^{\infty} \frac{\left(-|\alpha|^2\right)^m}{m!}\left|Fn(2\alpha \beta^*\psi^{2m-2}, \psi)\right|^2
	\end{aligned}
\end{equation}

В последнем ряду основной вклад вносят только члены c $n\sim|\alpha|^2 \pm 3|\alpha|$.
\footnote{Данную оценку можно сузить ещё сильнее, если учесть, что $Q(\dots)$ имеет гауссов колокол. Ширина этого колокола $\sqrt{|\alpha\beta|}2\Gamma$. Так как теоретически это значение порядка может быть велико и зависеть  от $\beta$, мы лучше в тупую просто просуммируем при всех соответсвующих $n$.}

Для дальнейшего разложения пригодится аналогичное уже упоминавшемуся разложение в Стирлинга:
\begin{equation}
    \ln\left(\frac{|\alpha|^{2m}}{m!}\right)\approx 2m \ln|\alpha| - \frac{1}{2}\ln m - m \left(\ln m - 1\right) - \frac{1}{12m}
\end{equation}
\end{document}