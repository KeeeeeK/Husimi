\documentclass[a4paper, 12pt]{article}

%Русский язык
\renewcommand{\familydefault}{\sfdefault}%шрифт
\usepackage[T2A]{fontenc} %кодировка
\usepackage[utf8]{inputenc} %кодировка исходного кода
\usepackage[english,russian]{babel} %локализация и переносы
%отступы 
\usepackage[left=2cm,right=2cm,top=2cm,bottom=3cm,bindingoffset=0cm]{geometry}
\usepackage{indentfirst}
%Вставка картинок
\usepackage{graphicx}
\graphicspath{}
\DeclareGraphicsExtensions{.pdf,.png,.jpg, .jpeg}

%Таблицы
\usepackage[table,xcdraw]{xcolor}
\usepackage{booktabs}


%Математика
\usepackage{amsmath, amsfonts, amssymb, amsthm, mathtools }
%Заголовок
\author{Нугманов Булат}

\begin{document}
\section*{Приближение для ряда}
Обозначение:
\begin{equation}
    \begin{aligned}
        \alpha = r e^{i\phi}
    \end{aligned}
\end{equation}

Стирлинг:
\begin{equation}
    \begin{aligned}
        n! \approx \sqrt{2\pi n} \left(\frac{n}{e}\right)^n e^{\frac{1}{12n}+\dots}
    \end{aligned}
\end{equation}

Логарифм одного слагаемого в $F-normalized$:
\begin{equation}
\begin{aligned}
    &\ln\left(\frac{\alpha^n e^{i\phi n + i\gamma n(n+1)}}{n!}e^{-r}\right)  \approx 
    \ln\left(\frac{r^n e^{i\phi n + i\gamma n(n+1)}}{\sqrt{2\pi n} \left(\frac{n}{e}\right)^n e^{\frac{1}{12n}}}e^{-r}\right) =\\
    &=n \left(\ln r + i \phi + i \gamma (n+1)\right) - \frac{1}{2}\ln\left(2\pi\right) 
    - \frac{1}{2}\ln n - n \left(\ln n - 1\right) - \frac{1}{12n} - r\\
    &=n-r + n\cdot\ln\left(\frac{r}{n}\right) + i n \left(\phi + \gamma(n+1)\right)
     -  \frac{1}{12n} - \frac{1}{2}\ln n - \frac{1}{2}\ln\left(2\pi\right) 
\end{aligned}
\end{equation}
\end{document}